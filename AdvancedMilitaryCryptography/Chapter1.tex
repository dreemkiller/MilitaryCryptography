\chapter{TRANSPOSITION SYSTEMS}

 

\section{MONOPHASE TRANSPOSITIOIN SYSTEMS}

\subsection{Transp05ition Systems Employing Geometric Designs}

In part one brief mention was made of the use of geometric designs
and figures other than rectangles in producing transposition ciphers. It
was stated that triangles, trapezoids, and polygons of various symmetrical
shapes can be employed. Figures of these types form connecting links
between the methods that use simple rectangular designs and the more
complicated methods that use figures in which transposition takes place
along diagonals.

\subsectionTrapezoidal Designs}

(1-. A trapezoid or, more accurately, a truncated triangle, of pre—
arranged dimensions as regards the number of cells (which in this case
are rhombs into which it is to be partitioned, is constructed. There will
be left on one side of the design a series of small triangles which are not
to be used for inscribing letters, and are therefore crossed off in the
design, as shown in figure 24. Only two agreements are necessary in
order .to fix the dimensions of the design: a keyword or keyphrase to
determine the number of cells at the base of the design, and an under-
standing as to the height of the design expressed in number of cells. The
successive horizontal rows of cells will decrease by one in number from
bottom to top of the design. In figure 24, the keyphrase NO CANDY
FOR ISSUE is used as a basis for deriving a numerical key of 15 ele—
ments, and it is assumed that by prearrangement it was agreed that the
height of the design should be eight cells. Therefore, the bottom row
has 15 cells, the next one upwards, 14, the next, 13, and so on, to the
last, with 8 cells. The inscription may follow any route agreed upon; in
the example, it follows the normal manner of writing. The transcription
follows the numerical key order, yielding this cryptogram:

87

 

 

 

 

 

REF 1D:A56932

ODAIK AEDME HPODV ITEIP NHUET BOBRO
HDTFS EISNI ETBEF BCBTM ESHGA RTORD
IRERE AWARR ERTNS IEPVR VASEO FTEDL
NA

b. Decryptographing is merely the reverse of cryptographing, there
being no difficulties provided that the design has been correctly con-
structed. For this purpose cross—section paper will be found useful. The
analysis of such a cryptogram is somewhat complicated by the presence
of columns having varying numbers of letters; it may be further com-
plicated by following complex routes in inscription. It is also possible

 

/’Ev71p’1y’D}rQ/'E/" Tim“
F I v E H U N DA13zmt
E 'D -D A s H L H A Sari
B R 0 K E N D o w n.43mm
T o P I P E R A T I vim,
E T H A T I T B E R E P A.mh
I R E D B E F 0 RA‘EEI§EIEEIaEIEWN
Aflflfiflflflflfl' IRSLM

7——9——2-1_-8-3-15--5—1o—11-—6—12-13-14.—4

'NOOANDYFORISSUE
Figur224.

to follow a numerical key in the inscription of the plain text in horizontal
lines; this additional procedure would further complicate and delay
solution.

87. Triangular Designs

a. The simplest way of drawing up a triangle for cryptographing is to
take cross—section paper, draw a square the side of which is equal to the
length agreed upon as expressed in the number of cells, and then draw
a diagonal cutting the large square into two equal triangles. This is
shown in figure 25, where the length agreed upon is nine, that is, nine
cells per side. The letters of the plain text are inscribed in accord-
ance with any prearranged route, the one illustrated in figure 26 be-
ing a simple method wherein the letters are inscribed in horizontal
lines in the normal manner. When so inscribed, the letters in the dia-
gram will form 2n — 1 columns where n is the number of cells
forming one of the sides of the square from which the triange has been
constructed. The total number of letters that can be inscribed within

 

 

REF ID:A56932

 

Figure 25.

thetriangleis the sum ofn + (n — 1) + (n — 2 + (n —— 3) +...
+ 1. For a triangle based upon a side of 9 cells, the sum is 9 + 8 + 7
+ 6 + 5 + 4 + 3 + 2 +1: 45. The letters may then be tran-
scribed to form the cryptogram by following another route, or by follow-
ing a derived numerical key applied to the base of the triangle. A simple
method of deriving a key of Zn — 1 elements from a key of n elements
or letters is exemplified herewith. Let the key be DIAGONALS, a word
of nine letters. Extend this key to Zn — 1 places by repetition, and then
assign numerical values as usual:

I! = 9; 2 n —- 1 = 17
O N A L S D I A G O N A L
5-13-2-11-17——6-10—3—8-16-l4——4-12

This numerical key is the one that has been employed in enciphering
the message in Figure 26.

 

 

5--9--1--7-15-13--2-1L-17-6-10--3--8-16-14--4-12
Cryptogram:

RICRC OCSGE DOONI UAOOE
SEYID RTISS DTSNR AUNTN
PERTR

Figure 26.

8‘2

 

fiver—'M -"-"" ' ' " '

REF ID:A56932

 

 

Cryptogram :

UUSOC YNTSO REOYS ONRER
DRITI DTOGD RANEO RICSN
CTRNI GENNE ATGSR OSIIR
SOIET RTUAI POECO TNESS
DPRCD AURSD

Figure 27.

b. By a slight change in procedure it is possible to encipher a mes—
sage and produce a text which, for the sake of accuracy in special
cases, is double the original length, but which is self—checking. Sup—
pose that instead of applying a single numerical key to the base of the
triangle, a double—length key is applied to the legs, as shown in
figure 27. Here the key is TRIANGLES, extended to double length
by simple repetition, as follows:

1—2-3-4-—5-6-7-8—9-10-11-12-13-14-15-16-17-18

Keyword: TRIANGLESTRIANGLES

Numerical key: 17-13-7-1-11-5-9-3-15-18-14—8—2-12—6-10—4-16
This key is applied to the legs of the triangle beginning at the lower
left-hand corner. The transcription then follows key—number order,
which results in doubling the length of the message but the repeated
letters are scattered throughout the whole message. In decryptographing
such a message the clerk merely omits the second occurrence of a letter
if it agrees (in identity) with its first appearance in the text.

6. Many variations in inscription and transcription can be employed
in the case of triangles as well as trapezoids. Some of the variations in
the case of triangles are shown in figure 28.

88. Diagonal Methods

a. A method involving diagonal transposition which is reported to
have been employed by the French Army in World War I is now to
be described. A numerical key is derived from a fairly long word or
phrase, and a rectangle is constructed, as in figure 29. The text is
inscribed in this rectangle in normal fashion, nulls being employed, if
necessary, to complete the last line of the rectangle.

90

 

 

 

REF ID:A56932

n . -. -' '.| I . ." ' \..-.w.=-v$whhfl?flllllll

 

 

17-13-—7--1—11—-5--9--3-15-14--8--2—12--6-10--I.-16

Inscription: Up left side, down right, alternately.

Transcription: (a) In rows from the base line, left to right and right to left,
alternately, upwards :

PISOS RNATU SIERS Etc.
(1)) In diagonals from right leg, in key—number order:
RIEDR OUAYN etc.
(c) In rows from left leg, in key-number order:
CTGEO YTCEU etc.
(d) From columns in key-number order:
CNROI TUGRU etc.

Figure 28.

Message: ENEMY BATTERY LOCATED AT WOODS 1,000 YARDS
SOUTHEAST OF MUMMASBURG HEAVY ARTILLERY
STOP THEY ARE FIRING AT RATE OF THREE ROUNDS
PER MINUTE FOR THE BATTERY X WILLS, MAJ.

Keyphrase: MIDNIGHT RIDE OF PAUL REVERE.

Enciphering diagram:

MIDNIGHTRIDEOFPAULREVERE
l5—11—2-16—12-9—10—22—19-13—3—4—17—8—18—1—23—14—20—5—24—6—21—7

 

E N|§| M YB A T T EE’IY Lo clfAJ T E D|§| TIWI 00
D [310 N ET H o u SM! DY AR [E] s so 1le HE
{El ST 0 PM U M M IEISB IEIR H E III VIZ! MEI TI
L LE R YS T o |E|THE TE E F 1131 NC |_‘A_'|T
R AT E or T [E] R EER ou lfiln s [E] ElEl MI NIQI
T er 0 RT |El E B ATT ER TIXIIE I LL [SIM AJ
Cryptogram:

ADARR SESAR NUANX YAAPH HAURA UWYFW
RHEDO TETFS HETBE RTOIL TGIMO EITJO
YRURB TMSFT AHUTT NSLAE YEFYO RESTE
AESII EDLRT MNORE OLDYO ECAGR YTUMR
BDSVE LOHTN ATOMO ETEFS TANM

Figure 29.

91

 

REF ID:A56932

b. The correspondents agree beforehand upon several diagonals which
run from left to right, and from right to left and which intersect,
thus cutting up the design quite thoroughly. In figure 29 let these selected
diagonals be those indicated by the numbers from 1 to 6, inclusive,
the odd ones indicating diagonals running from left to right. In the
transcription, the letters along the indicated diagonals are first set down
in groups of five, proceeding in key—number order. Correspondents must
also agree beforehand as to whether a letter which lies at the intersection
of two diagonals will be taken both times it is encountered or taken
only once and, if so, whether on its first or second appearance. After
all these letters have been written down, one then proceeds with the
remaining letters in the usual columnar manner, omitting the letters
which have already been taken. The cryptographing process will become
clear upon the study of the example in figure 29.

89. Interrupted Keyword Transposition

a. This method of transposition is a development of a more simple
method wherein the transposition follows a numerical key. The latter
must first be described. A keyword or keyphrase of fair length is selected
and a numerical key derived from it. Let this key be the phrase UNI-
FORMITY OF METHOD.

Keyphrase: UNIFORMITYOFME TH OD
Numerical key: 17-10-6-3-11-14-8-7-15-18-12-4-9-2-16-5-13-1
The plain text is then written out in horizontal lines corresponding
to the length of the key; then transposition is effected within each row,
according to the sequence of numbers applicable, as shown in figure 30.

Message: ADMINISTRATIVE ORDERS MUST BE COMPLETED AND
READY TO ACCOMPANY FIELD ORDERS NOT LATER
THAN 5:00 RM. THIS DATE.

Enciphering diagram :

 

I”, 17-10-6-3-11-14--871-5 18-12-4--92-16-5-15-1
A DMI N IST R A TIVE on DE
R suu s TBE c o MPLE TE DA
N DRE A DYT o A ccou PA NY
F IEL D 0RD E R sno'r LA TE
R THA N FIV E r MTHI SD Ar
E

Cryptogram:
EEIIR MTSVD NTDIR OAAAE UPEME BLSSM
DTG'I'R OYMEC ARTYO DACND OPNAE TLNAE
DROID STOEL FRTIA TDHVI HTNMA FESRP
E

Figure 30.

9.2

 

 

REF ID:A56932

b. In the foregoing case the eneipherment takes place only by trans—
position within rows, but it is possible to complicate the method by
transposing, in addition, the rows as a whole, employing the same key
or only a portion of it, as much as is required. Thus, if the message
contained 18 rows of 18 letters each, then the transposition of rows
could be effected according to key—number order, the last row being
taken first (since the number 1 of the numerical key happens in this
case to be at the end of the numerical key), the 14th row being taken
second (since the number 2 of the numerical key is the 14th number),
and so on. Where the message does not contain as many complete rows
as there are numbers in the key, the transposition takes place in key-
number order nevertheless, the rows being taken in the numerical order
of the numbers present. Using the same key and message as in the
foregoing case, the encipherment would be as shown in figure 31.

Eneiphering diagram:
l7-10-6-3-11-14-8-7-15-18-12-4-9-2-16-5-13-1

 

l7: ADMINISTRATIVE OR DE
10: RSMUSTBECOMPLE TE DA
6: NDREADYTOACCOM PA NY
3: FIELDORDERSNOT LA TE
11: RTHANFIVEPMTHI SD AT
14: E

Cryptogram:

ETLNA EDROI DSTOE LFRYM ECART YODAC
NDOPN AAEUP EMEBL SSMDT CTROT IATDH
VIHTN MAFES RPEEE IIRMT SVDN'I' DIROA
A

Figure 31

c. From the preceding method it is but a step to the method of
interrupted key transposition now to be described. Instead of writing
the text in regular-length groups corresponding to the length of the
key, it is written out in irregular groups the lengths of which vary
according to some prearranged plan. For example, note the basis of
the variable grouping in figure 32, which uses the same message and
key as in a above.

(1. This method may be combined with that shown in b above, thus
further complicating the system. In decryptographing such a message it
is best to use cross—section paper, block out the cells to be occupied by
letters in the deciphering diagram, and indicate the key numbers appli-
cable to each line. This will facilitate the process materially and help
eliminate errors.

93

' :nlififluli I||

 

REF ID:A56932

Enciphering diagram:
17-10—6—3- ll-14—8—7-15-18-12—4—9—2-16—5-15—1

 

A D M I N I S T R A T I V E 0 R D E
R S M U S T B E C 0 M P L E T E D A
N D R E A D Y T 0 A C C O M P A N Y
F I E L D 0 R D E R S N O T L A T E
R T H A N F I V E P M T H I S D A T
E

H
‘1

-10—6—5-11-l4—8—7-15-18-12—4—9—2-16—5-13—1

N STRATIVEORDE
S BECOMPLE

 

*iH

ADYTOACC......
NYFIELDORDER..

>FJ>CH

T E R T H
I V E P . . . . . . .
I S D A T E (L C E P)*.

(“The four final letters LCEP are nulls, to complete the row.)

=>oamoza=u>
D-JZP‘ZEUL‘JUIU
:qpowwuzs

Cryptogram (columnar transposition in key-number sequence):

EEEDI UAEAT IIIPC OERRM MDRPO AFHTE
TIHTS BYFTP AVLRP DSEDM NLNTN SANEV
STMCD CDITD YREDR COEEO EARTN OSTAM
AOALL

Figure 32.

e. Another method of interrupted transposition is that which employs
a rather long sequence of digits to control the interruption. In order
to avoid the necessity of carrying around such a written sequence, it
is possible to agree upon a number whose reciprocal when converted
by actual division into its equivalent decimal number will give a long
series of digits. For example, the reciprocal of 7, or 1/7, yields a
repeating sequence of six digits: 142857142857 . . .; the reciprocal
of 49, 1/49, yields a repeating sequence of 42 digits, etc. Zeros, when they
appear, are omitted from the sequence. Suppose the number 19 is agreed
upon, the reciprocal of which yields the sequence (0)52631578947368421.
On cross-section paper mark ofl' sets of cells corresponding in number
to the successive digits. Thus:

 

 

5 2 (i 8 l 5
||l|l|><|||><|||l||l><||l|><l|><|||l|l
Let the message be ATTACK HAS BEEN POSTPONED.
Encipherment :
5 2 6 3 1 5

 

willIEIsloIXITIll><l1rl$lNITINIDIXIAIBIPI><I¢I><IKIEI0IPITI
Cryptogram:
AHESO TATSN TNDAB PCKEO PE

94

 

:33 L V:

F‘ ETQ"E'§$'33’E

 

 

REF ID:A56932

' '--' ...,“w.'. ' ---:'=‘--.-."..'=-'-._ .....

 

f. To decryptograph such a message, the cryptogram is written down
in a series of cross-section cells, which are then blocked off in sets
according to the numerical key:

5 2 6 3 1 5

IAIHIEISIOIXITIAIXITISINITINIDIXIAIBIPIXICIXIKIElolPTfl

Taking the letters in consecutive order out of the successive sets, and
crossing them off the series at the same time as they are being written
down to construct the plain text, the message is found to begin with
the following two words:

5. 2 6 3 1 5
IAIHIEISIOIXITIAIXITISINITINIDIXIAIBIPIXICIXIKTEIOIPIEI
ATTACK HAS . . .

g. Preparatory to cryptographing, it is necessary to find the length
of the message to be enciphered and then to mark off as many cells as
will be required for encipherment. Nulls are used to fill in cells that
are not occupied after cnciphering the whole message. The secrecy of
the method depends, of course, upon the reciprocal selected, but there
is no reason why any fraction that will yield a long series of digits
cannot be employed. If the selection of key numbers were restricted
to reciprocals, the secrecy would be more limited in scope than is actually
necessitated by the method itself.

 

 

90. Permutation Method

0. An old method, known in literature as the aerial telegraphy method,1
forms the basis of this system. A set of permutations of 3 4, . . .
9 digits is agreed upon and these permutations are listed in a definite
series. As an example, let these permutations be made of the digits 1
to 5, selecting only four of the possible 120. Suppose those selected
are the following, set down in successive lines of the diagram in
figure 33a:

 

 

 

 

 

 

 

 

 

 

 

 

 

 

 

Permutation

2 3 1 5 4 2 3 1 5 4

3 2 5 1 4 3 2 5 1 4

1 5 5 2 4 1 5 3 2 4

4 3 1 5 2 4 3 1 5 2
Figure 33a.

1 So named because it was first devised and employed in messages transmitted by a system of

semaphore signaling in practical usage in Europe before the electrical telegraph was invented.

95

 

 

 

 

REF ID:A56932

The letters of the plain text, taken in sets of fives, are distributed
within the sections of the diagram in accordance with the permuta-
tions indicated above the sections and also at the left. Thus, the first
five letters of the text, supposing them to be the initial letters of. the
word RECOMMENDATIONS, are inserted in the following positions:

Permutation

 

 

23154

 

 

 

 

 

 

E C R M 0

 

The next five letters are inscribed in the second line of the diagram
in the sections indicated by the permutation above and at the- left of
the line. Thus:

 

 

 

 

Permutation
' 2 3 1 5 4
2 3 1 5 4 E C . R M O
3 2 5 1 4
l 4
5 2 5. N E A M D

 

 

 

 

 

 

 

This process is continued for each line and for as many lines as there
are permutations indicated at the left. In the foregoing case, after
twenty letters have been inserted, one inserts a second set of five
letters again on the first line, placing the letters of this second set
immediately to the right of those of the first set, respectively in key-
number order. The succeeding lines are treated in similar fashion
until the whole message has been enciphered. The following example
will illustrate the process:
Message: RECOMMENDATIONS FOR LOCATION OF NEW
BALLOON POSITIONS MUST BE SUBMITTED
BEFORE 12TH AIRDROME COMPANY CHANGES
COMMAND POST TOMORROW.

Enciphering diagram :

 

 

 

 

 

 

 

 

Permutation
2 .3 1 5 4 2 3 1 5 4
EASEOM CTIDMA RCOTRM MOIECD OITBEN
3 2 5 1 4 5 2 5 1 4
_ NOSRPS ESNOMO ANUTNT MNOFOP DF'MEAT
1 5 3 2 4 1 5 5 2 4
TESWYO 'SLSTNR OBBLHO IWTECM NAEFAR
4 3 1 5 2 4 3 1 5 2
LNIRCB* ROMISC* FLUHGO op'mow OOBAEW

 

 

 

 

 

 

 

' The letters B. G. and D are nulls. to complete the figure.
' ' " Figure 3317.

‘96

 

Fun“- ¢.__ ..

 

REF ID:A56932

The letters of the cipher text are taken from the diagram according
to any prearranged route, the most simple being to transcribe the lines
of letters in groups of fives, thus:

EASEO MCTID MARCO TRMNO IECDO ITBEN
NOSRP SESNO MOANU TNTMN OFOPD FMEAT
TESWY OSLST NROBB LHOIW TECMN AEFAR
LNIRC BROME SCFLU HGOOP TDODO OBAEW

b. The foregoing method when employed in its most simple form
does not yield cryptograms of even a moderate degree of security;
but if the method of inscription and transcription is varied and made
more. complex, the degree of security may be increased quite notice-
ably. It is possible to use longer permutations, based on sets of 6,
7, 8, or 9 digits, but in every case the successive permutations must
be prearranged as regards both their exact composition and their order
or arrangement in the diagram. '

91. Transposition Method Using Special Figures

0. The method now to be described is useful only in special cases
where the correspondence is restricted to brief communications between
a very limited number of persons. It is necessary to agree in advance on
certain particulars, as will be seen. Let the message to be enciphered
be the following:

FOUR TRANSPORTS WILL BE COMPLETED BY END
OF APRIL AND SIX MORE BY END OF JULY.

Note the following figures and enciphcrment:

 

 

 

 

 

 

o n r s 1. o
/”- N /" I l I i -
r u T——A s o r w r. n c u

L/ 4 l l i

n / n n 1 r. r

I n n P A 1
L——J—r n-———-r n+0 A——R n+1! 3+1:

1: r 1 n u

r 1'
c n r n 0—-l-—-J’ 1.___..._

n n ' u
Cryptogram:

ORPSL OFUTA SOTWL BCMRN RIEPE BDPAI
LTDYN OARLN SXEEF IDMRE FYOEY NOJLB
DU

Figure 34.

97

REF ID:A56932

b. It will be noted that it is essential to agree in advance not only
upon the nature of the figure but also upon the number of figures per
line.

c. The next series is a modification of the preceding. The same

i message will be employed, with a double-cross figure, five figures per

 

 

 

 

 

 

 

 

 

 

 

 

 

 

 

 

 

 

 

 

 

 

line.
\ o u r o 1. a a a n c
S D E ' I 9
l I 2 C I D I A
A II. I S M 0 ‘l 3 It I’
A I I I J
D n V
Y 1-
a x 1 I s v
| Cryptogram:

ouror. BETDO FRSRL ELI-INF NTITP CEDIA
| ARwsu oven? ANREF JLDOB owsn mm
EY

l Figure 35.

(1. Still another series may be formed, as follows:

 

 

 

 

 

 

 

 

 

F s I. I N
l u——-—-o I———P r—-—I. n—-—-E I———D
l 2““; 3“: Li ;__§ 33:2
| 1' r c n A
l I- c c
§:*t 3:3¢ :“§
1‘51» a~—3 Y_:u
s 1 1.
Cryptogram:

FSLLN NOIPP LEEID AUWOM BYTRO RRSRO
EBEPF TTCDA LOOMA DRFXN NEJID EBYUS
YL

Figure 36.

e. A figure of different form than the preceding forms the basis of
the next type.

 

98

1-"

REF ID:A56932

OOEDRTOYRWPNNLE

rpazuncnrsmrixnnsn'r

FNROPSBJ’IXEL

OAODADEFRIYULMN!

Cryptogram:
OOEDR TOYRW PNNLE FPBEU RCBTS

MEAIL DSLTF NROPS BJIXE LOAOD
ADEF‘R IYULM NY

Figure 37.

f. From the foregoing examples, it is obvious that many other figures
may be used for effective transpositions of this kind, such as stars
of varying numbers of points, polygons of various symmetrical shapes,
etc. It is merely necessary to agree upon the figures, the number of
figures per line, the starting points of the inscription and transcription
processes.

9. The method lends itself readily to combination with simple
monoalphabetic substitution, yielding cryptograms of a rather high
degree of security.

Section II. POLYPHASE TR'A-NSPOSITION SYSTEMS

92. Polyphase Transposition Methods in General

a. In paragraph 33, brief mention was made of transposition systems
in which two or more processes of rearrangement are involved. It was
stated that only a very limited number of such transposition methods
are practicable for military use, but that the degree of security afforded
by them is considerably greater than that afforded by certain much
more complicated substitution methods. The methods referred to are
those which involve two or more successive transpositions, and merely
for purposes of brevity in reference they will here be called polyphase
transposition methods to distinguish them from the single monophase
methods thus far described.

b. It is obvious that a polyphase transposition method may involve
2, 3, . . . successive transpositions of the letters of the plain text.
To describe these methods in general terms, one may indicate that
the letters resulting from a first transposition, designated as the T—l

99

REF ID:A56932

transposition, form the basis of a second, or T—2 transposition. If the
process is continued, there may be T—3, T—4 . . . transpositions, and
each may involve the use of a geometric figure or design. For con—
venience, the design involved in accomplishing the T—1 transposition
may be designated as the D—1 design; that involved in accomplishing
the T—2 transposition as the D—2 design, etc. However, it may as well
be stated at this point, that so far as military cryptography is concerned,
methods which involve more than D—2 and T—2 elements are entirely
impractical and often those which involve no more than D—2 and T—2
elements are also impracticable for such use.

93. True and False Polyphase Transpositions

a. It is possible to perform two or more transpositions with the
letters of a text and yet the final cryptogram will be no more difficult
to solve than if only a single transposition had been effected. The equiva-
lent of this in the case of substitution ciphers is to encipher a mono-
alphabetic cryptogram by means of a second single alphabet; the final
result is still a monoalphabetic substitution cipher. Likewise, if a mes-
sage had been enciphered by a simple form of route transposition
and a second and similar or approximately similar form of simple
route transposition is again applied to the text of the first transposition,
the final text is still that of a monophase transposition cipher. Again,
two transpositions may be accomplished without really affecting a more
thorough scrambling of the letters composing the original text. Examples
will serve to clarify the differences between false and true polyphase
transposition.

b. Note the following simple columnar transposition cipher pre-
pared according to the method described in paragraph 27 :

Message: DELIVER ALL AMMUNITION TO 4TH DIVISION
DUMP-

Keyword : SCHEDULE = i:

q

SCHEDUL
-1-5-5-2-8-6-

Enciphering rectangle :

I—l

 

 

 

 

 

vac-arcs:
HWHI‘H

oao>rcn
23:23:49:
Dua=<m
:Hocmm
a<wzwa>
‘UHOHbIDh

 

 

 

 

 

 

 

 

 

 

Cryptogram (T—l):

ELI'RI VMTDD IMNHN A I
DL'PUS EUOIU IO P LAOTO mm

Figure 38.
1 00

 

__ 7‘ _,.-., .Pm-mw—uum

REF ID:A56932

._.___.... mum—s- mai'!'-;l:i""l"lih l... ‘ “L'-

' “III“ a:

In producing the foregoing cryptogram only the columns were trans—
posed. Suppose that by prearrangement, using the keyword BREAK
(derived numerical key = 2—5—3—1—4), the horizontal lines of the fore—
going enciphering rectangle were also to be transposed. For example,
let the horizontal lines of the rectangle D—l be transposed immediately
before taking the letters out of the columns of the design (in key-number

--- 'NV‘F’E‘EWH

wUW-fl ra-r'zr -=—~1e--'1=r*r”'. w

 

order) to form the cipher text. Thus:

.—

 

 

 

 

 

 

ne- .. my: to
In c: all" U
L" m a Uc.‘ «1'
>0 O l"!-] 0‘
32 2 H 3: ca

 

HWHr'm
on—Jo:>r'
ZIZEH
CHOCFJ
S<'¢12>§1
’UHOH>
r'H'Hmw
EUH<UI~7
CGOHHco
Z:".15U<G=
H-uo>H~h

 

 

 

 

 

 

 

 

 

 

 

 

 

 

 

 

 

D— l
Cryptogram (T—2):

REIIL DVTDM HINNM IAOPI TLOOA VRFMN
UDTSL IEOUU

Figure 39.

c. The foregoing, however, is not a case of true polyphase or- so-
called double transposition. The same final result may be accomplished
in a way which will at first glance appear quite different but is in.
reality one that accomplishes the same two operations by combining
them in one operation. Let the message be inscribed as before, but this
time with both numerical keys applied to the top and side of the
rectangle. Then let another rectangle of the same dimensions, but with
numbers in straight sequence instead of key—number sequence, be set
alongside it. Thus: ' '

7

p.-

5

 

 

 

 

 

 

 

 

 

 

 

 

 

 

 

 

3 2 8 6 4
2 D E L I V E R A 1
5 L L A M M U ' N I 2
3 T I 0 N T 0 F 0 3
1 U R T H D I V I 4
4 S I 0 N D U M P 5
D—l
Figure 40.

Each letter D—l is now transferred to that cell in D—2 which is indicated
by the row and column indicators of the letter in D—l. For example,
the first letter, D, of D—l, has the indicators 2—7 and it is placed in

101

 

REF ID:A56932

the 2—7 cell in D—Z; the second letter of D—l, which is E, is placed
in the 2—1 cell of D—2, and so on. The final result is as follows:

 

 

 

 

 

 

 

 

 

 

 

 

 

 

 

 

 

 

 

 

 

 

 

 

 

 

 

 

 

 

 

71532864 12345 rs
ans I'VERA 1R -HITVUI
I 5LLAMMUN12EVIALRDE
. arrourorosrruoorro

1URTHDIVI4IDNPOMSU

4SIONDUMP5LMMIANLU
I 13-1 . D—2
Figure41.

It will be seen that if the columns of D—2 are now read downwards
in straight order from left to right the final cryptogram is identical
with that obtained in figure 39: REIIL DVTDM, etc.

d. The foregoing cipher, often called the Nihilist Cipher, is referred
to in some of the older literature as a double transposition cipher
becauSe it involves a transposition of both columns and rows; and
I. indeed as described in b above it seems to involve a double process.

It is, however, not an example of true double transposition. When

the mechanism of this cipher is compared with that now to be
. described, the great difference in the cryptographic security of the two
. methods will become apparent.

94. True Double Transposition

In the form of the false double transposition described above, it is
only entire columns and entire rows that are transposed. The disarrange-
ment of the letters is after all not very thorough. In true double trans-

; position this is no longer the case, for here the letters of columns and
i rows become so thoroughly rearranged that the final text presents a
complete scrambling almost as though the letters of the message had
been tossed into a hat and then drawn out at random.

i Section I". TRUE DOUBLE TRANSPOSITION
i 95. True Double Transposition of the Columnar Type

a. It is by what is apparently a simple modification of certain of the
c0111mnar methods already described that an exceedingly good true
double transposition can be effected. Let a numerical key be derived
from a keyword in the usual manner and let the message be written
out under this key to form a rectangle in the usual manner for colum-
nar transposition. The length of the message itself determines the
anct dimensions of the rectangle thus formed, and whether or not it
‘5 completely or incompletely filled.

102

 

 

 

 

REF ID:A56932

b. In its most effective form the double transposition is based upon
an incompletely filled rectangle; that is, one in which one or more cells
in the last line remain unfilled. An example of the method now follows:
Let the keyword be INTERNATIONAL; the message to be enciphered,

as follows :

OUR ATTACK SLOWING UP IN FRONT OF HILL 1000 YARDS
SOUTHEAST OF GOLDENVILIE STOP REQUEST PROMPT

REEN FORCEMEN T.

Keyword :

IN TE RNA TI ONAL

Derived numerical key: 4-7-12-5-11—8-1-15—5-10-9-2-6

4——‘7—12—3~—11—8——.1—13-5— 10 —9.—-2-~6

 

U

R

A

T

A C

K

S

0.

 

 

 

 

 

 

want-imam

MMUOL‘EO

PJDHCOZL"

ZCZHCD-J

fim<=mo=

 

OMHMU’WHO

 

 

NSF>ZIZ

 

Q’UE‘MUHQ

 

resume-4r:

 

zomo>rm

 

NKHWWOHI-l
FED-J'UOUIIFI"!

 

 

 

 

 

 

 

 

 

4—7—12 —3—11 — 8— 1—13 — 5- 10 — 9—2— 6

 

A

N

N

D

G

OPN

0

T

U

T

N

 

U

N

 

 

 

 

 

 

 

 

 

 

 

 

 

 

 

 

 

 

 

 

 

D-2
Figure 42a.

103

REF ID:A56932

The first, Or D—l, rectangle is inscribed in the usual manner of simple
numerical key columnar transposition. It is shown as D—l in the accom-
panying figure. . The letters of T—l transposition are then inscribed
in the second, or D—Z, rectangle in the normal manner of writing, that
is, frorn left to right and from the top downwards. This is shown in
D—Z of figure 42a for the first two columns of D—l (in numerical key
order.) after transfer of their letters into D—Z. The letters of the
remaining columns of D—1 are transferred in the same manner into
D—Z, yielding the following rectangle:

4—7—12-3—11 — 8—- 1—13— 5- 10 — 9—2—6
N'NgDGOPNOTUT

 

 

 

 

 

 

 

UHMIMW
mrar'zr'm
I‘D-12>5UO
'U'UOIT‘WH
OF‘CHE'TJZ

 

I c: :> m in Q > c: .:>
'11 o o a m :21 z
ta tn ta H :x: H :>.
m” o m o < m c:
o g o 5:: [=1 0 r'
*u so :1: '21 u: z}: -<
a o o a c: an 2-1
-a H U a z v—a I351

 

 

 

 

 

 

 

 

 

 

 

 

 

 

 

Figure 421).

For the T—Z text- the letters are transcribed from the D—2 rectangle,
reading down the'columns in key-number order, and grouping the letters
in fives. The Cryptogram is as follows:

PTRUT OGTTI RLOPP DUSVO SOSAU AOREA
CORSH EEDNF WTULC NNEST QOFOY KFFHR
PUORA NTLTE LNLES GLOER OMONA IHIES
ENETN MDIT

e. In paragraph 29 a variation of the simple columnar key method
of transposition was described. If the process therein indicated is
repeated, double transposition is effected. The following example will
serve to illustrate the method, using the same message and key as were
used in the paragraph 29:

Message: REQUEST IMMEDIATE REENFORCEMENTS
Keyword: P R 0 D U C T
Derived numerical key: 4-5-3-2-7-1-6

:104

 

 

REF 1D:A56932

Encipherment :

4-5—3-2—7—1-6 4-5—5-2-7-1-6 4—5-3—2—7-1-6
TextzREQUESTIMMEDIATEREENF"
T—l:SINEUEEEQMRCRITOTEMER
T—2:EREEEREFNMTASETSEIQOT

4—5-3-2—7-1—6 4—5

0 R C E M E N T S

S 1' A F N E D E M

M E I R D U C M N
Cryptogram:

EREEE REFNM TASET SEIQO TMEIR

DUCMN

d. In some respects this modified method is simpler for the novice-to
perform correctly than is that employing rectangles. Experience has
shown that many inexpert cryptographic clerks fail to perform the two
transpositions correctly when D—1 and D—2 rectangles are employed
in the work.

96. General Remarks on True Polyphase Transposition

a. The cryptographic security of the true double transposition method
deserves-discussion. Careful study of a .cryptogram': enciphered by; the
double transposition method set forth in paragraph 95 b and c will
indicate that an extremely thorough scrambling of the letters is. indeed
brought about by the method. Basically, its principle is the splitting up
of the adjacent or successive letters constituting the plain text by two
sets of‘ cuts” , the second of which IS in a direction that 15 perpendicular
to the first, with the individual “cuts” of both sets arranged in a
variable and irregular order. It is well adapted for a regular and
voluminous exchange of cryptograms between correspondents, because
even if many messages in the same key are intercepted, so" long .asllno
two messages are identical in length, they can only be cryptanalyzed
after considerable effort.

b. Triple and quadruple transpositions of the same nature are possible
but not practical for serious usage. Theoretically, a continuation or
repetition of the tranposition process will ultimately bring about a condi-
tion'iivherein the D-n rectangle is identical with the D'~1. rectangle: in
other words, after a certain number of transpositions the rectangle pro—
duced by a repetition of the cryptographing process results finally in
decryptographing the message. Exactly how many repetitive transposi-
tions intervene in such cases is extremely variable and depends upon
factors lying outside the scope of this text, I I

1:195

REF ID:A56932

c. In the example of cryptographing given in paragraph 95b, the
D—1 and D—Z rectangles are identical in dimensions, and identical
numerical keys are applied to effect the T—l and T—2 transpositions.
It is obvious, however, that it is not necessary to maintain these identi—
ties; D—1 and D—2 rectangles of different dimensions may readily be
employed, and even if it is agreed to have the dimensions identical, the
numerical keys for the two transpositions may be different. Furthermore,
it is possible to add other variable elements. (1) The direction or manner
of inscribing the letters in the D—1 rectangle may be varied; (2) the
direction of reading off or taking the letters out of the D—1 rectangle
in effecting the T—l transposition, that is, in transferring them into the
D—2 rectangle, may be varied; (3) the direction of inscribing these
letters in the D—2 rectangle may be varied; ( 4) the direction of reading
off or taking the letters out of the D—2 rectangle in effecting the T—Z
transposition may be varied.

d. The solution of cryptograms enciphered upon the double transposi-
tion principle is often made possible by the presence of certain plain—text
combinations, such as QU and CH (in German). For this reason, care-
ful cryptographers substitute a single letter for such combinations, as
decided upon by preagreement. For example, in one case the letter Q
was invariably .used as a substitute for the compound CH, with good
effect.

Section IV. GRILLES AND OTHER TYPES 'OF MATRICES

97. Type of Cryptographic Grilles

Broadly speaking, cryptographic grilles2 are sheets of paper, card-
board, or thin metal in which perforations have been made for the
uncovering of spaces in which letters (or groups of letters, syllables,
entire words) may be written on another sheet of paper upon which the
grille is superimposed. This latter sheet, usually made also of cross-
section paper, will hereafter be designated for purposes of brevity in
reference as the grille grid, or grid. Its external dimensions are the
same as those of the grille. Grilles are of several types depending upon
their construction and manner of employment. They will be treated here
under the titles of (1) simple grilles, (2) revolving grilles, (3) non-
perforated grilles, and (4) “post card” grilles.

98. Simple Grilles

a. These consist usually of a square in which holes or apertures have
been cut in prearranged positions. When the grille is superimposed upon
3Also often called “stencils." The general term matrix (plural, matrices) is very useful in

referring to a geometric figure or diagram used for transposition purposes. Other terms in
common use are cage, frame, bar, etc.

106

 

REF ID:A56932

the grid, these apertures disclose cells on the grid, in which cells letters,
groups of letters, syllables, or entire words may be inscribed. An example
is shown in figure 43. The four sides of the obverse surface of the grille
are designated by the figures 1, 2, 3, 4; the four sides of the reverse
surface, by the figures 5, 6, 7, 8. These figures are employed to indicate
the position of the grille upon the grid in encipherment.
b. (1) In cryptographing a message the grille is placed upon the grid,
in one of the eight possible positions: Obverse surface up, with
figure 1, 2, 3, or 4 at the top left; or reverse surface up, with

 

 

 

 

 

 

 

 

 

 

 

 

 

 

 

 

 

 

 

 

 

 

 

 

 

 

 

 

 

 

 

 

 

 

 

 

 

 

 

 

 

 

1
57 %’//777 V%*
7%%,//////7/7A ”/A
/ fifil/A/AV/ffl/
Wf/é/O/ @fifi/JV
éf/AV/A /%7/ 0:0;
Q/AVA @f/A A202.
0074/ [A A W47
QWVJVA 7/ @717
75% 7A Af///V/fl
u A // //// e
(5)
27A AV/V/Vf . 7/ 3':
QVA W/fflO/AVAV/zy
0/57/71 /,7 7A
%7/47//A5 @9047
@714 7/71/ ’Af/fl/
/%0%%é 7// V
90 4% T/// QV
VAW/V/z 7/, Ay/zy/V
, @QVAV/ OVA
SEX V/V/ ///’u.)
Figure 43.

107

 

1%

REF ID:A56932

figure 5, 6, 7, or 8 at the top left. The letters of the plain text
are then inscribed in the cells disclosed by the apertures, follow—
ing any prearranged route. In figure 44, the normal manner of
writing, from left to right, and from the top downwards, has
been followed in the inscription, the message being ALL
DESTROYERS OUTSIDE.

g5)

 

[7/

7,

// .
E

74

74

 

\
(8) 7'

 

A /
/%
11%
7A
QT

R
@WAYVAE
7&8 OV/X/X/A
7/8 I //

 

 

 

V
\\
\ m§xx
V

k\\
e §s\\‘§&&§\\\“

 

 

 

 

'* =° O\x§s\\\lr‘ \\ t‘
®S§\ ®§§QQ\

 

 

 

 

 

 

 

 

 

 

 

2%

 

(L)€

Figure 44.

(2) The transcription process now follows. The cipher text is

written down, the letters being taken by following any pre-

arranged route, which must be perpendicular to the route of

inscription, otherwise the letters will follow in plain-text order.

In the following, the route is by columns from left to right.
Cryptogram:

LRTAD TSSER YOIDS ELOEU

(3) If the number of letters of the plain—text message exceeds the

number of cells disclosed by one placement of the grille, the
letters given by this placement are written down (in crypto-
graphic order), and then the grille is placed in the next position
on a fresh grid; the process is continued in this manner until
the entire message has been cryptographed. The several sections
of the cipher letters resulting from the placements of the grille
on successive grids merely follow each other in the final crypto-
gram. In this manner of employment it is only necessary for
the correspondents to agree upon the initial position of the grille
and its successive positions or placements.

REF ID:A56932

c. It is obvious that by the use of a simple grille the. letters of a
message to be CTYPtngaPhed may be distributed within an enveloping
message consisting mostly of “dummy” text, inserted for purposes of
enabling the message to escape suppression in censorship. 'For example,
suppose the grille shown in figure 43 is employed in position 1 and the
message to be conveyed is ALL DESTROYERS OUTSIDE. The
letters of this message are inscribed in their proper places on the grid,
exactly as shown in figure 44. An “open” or disguising text is now to
be composed; the latter serving as an envelope or “cover” for the letters
of the secret text, which remain in the positions in which they fall on
the grid. The open or disguising text, in other words, is built around or
superimposed on the secret text. Note how this is done in figure 45, with
an apparently innocent message reading:

I HAVE WORKED VERY WELL ALL DAY, TRYING TO- GET
EVERYTHING STRAIGHTENED UP BEFORE GOING ON MY
NEXT TRIP SOUTH, BUT INSIDE TEN DAYS . . .

 

 

 

 

 

 

 

 

 

 

 

 

 

 

 

 

 

 

 

 

 

 

1(5)

‘IHAVEWORKEg.

DVERYWELLAg
LLDAYTRY-IN
.GTOGETEVER
"YTHINGSTRA
.IGHTENEDUP
BEFOREGOIN
GONMYNEXTT
G-RIPSOUTHBU
«TINSIDETEN
Figure 45. (1.) g

d. The foregoing method naturally requires the transmission of con—
siderably more text than is actually necessary for conveying the message
intended. Where questions of censorship are not involved, the method
is therefore impractical. A modification of the method suggests itself in
the use of a transparent sheet of paper superimposed upon a square or
other figure in which the individual cells are irregularly numbered and
the inscription process follows the sequence of numbers. An example is
shown in figure 46, using the message ROCK CREEK BRIDGE WILL
BE DESTROYED WHEN TAIL HAS CROSSED.

109

 

REF ID:A56932

 

21.3944 715
372941 1114531
184310 24 20 28 14
12 84248 43338
354730 462617
191332224036 9

OI
N
0|

 

 

 

 

 

Sugar»;
cognac:
FHHFHO
crummy-a
ICON”!!!
coat/awn:
>mxmm°
>WH‘MH
fiHr‘n‘H

 

 

 

 

 

 

 

 

 

 

 

 

 

 

 

 

 

 

 

Figure 46.

The transcription may now follow any prearranged route. The normal
method of reading would produce the cryptogram beginning WCTEH
OEERI, etc. It is obvious that the correspondents must possess designs
with identically numbered cells.8

99. Revolving Grilles

a. In this type of grille (see fig. 47a) the apertures are also formed
by perforating a sheet of cross-section paper according to prearrange-
ment, but these apertures are so distributed that when the grille is
turned four times successively through angles of 90° and set in four
grille positions on the grid, all the cells on the grid are disclosed in turn.
(The preparation of such grilles is discussed in par. 103.) If letters are
inserted in the cells so disclosed, then after a complete revolution of the
grille every one of the cells of the grid will contain a letter and thus the
grid will be completely filled. For this reason such a grille is also called
a self—filling, or an automatic-completion grille. The secrecy of messages
enciphered by its means is dependent upon the distribution or position of
the apertures, the sequence of grille positions on the grid, that is, whether
in the order 1, 2, 3, 4 clockwise; or 1, 3, 4, 2 etc.), and the route followed
in inscribing and transcribing the letters in the cells of the grid. For each
position of the grille, one-fourth the total number of letters of the text
is inscribed; hence it is convenient to refer to “sections” of the text, it
being understood that each section consists of one-fourth the total num-
ber of letters.

b. There are two possible procedures so far as the inscription-trans-
scription sequence is concerned. (1) The letters of the plain text may be
inscribed in the cells of the grid through the apertures disclosed by the
grille and then, when the grid has been completely filled, the grille
removed, and the letters transcribed from the grid according to a pre-
arranged route; or, (2) the letters of the plain text may first be inscribed
in the cells of the grid according to a prearranged route and then the
grille applied to the completely-filled grid to give the sequence of letters

'The system employed by the French Army in 1886 was of the nature here described.

110

 

 

REF ID:A56932

Cryptogram :

 

 

LHICV YROOT WILHN F'SOMT
HURTI TCU'LO ROEDA TMVUI
ESTEL YF'RMU RNSF'E FASES
ESEAT OIDTL YNOIN AHEAH I
EDFOT NHSHH ETAMI YOSRE

Figure 47.

111

BM\\\1\\\\}\\\\!\\\\I

 

REF ID:A56932

forming the cipher text of the transcription process. The first method
will be described in c below; the second in e below.

c. Taking the simplest manner of inscribing the letters, that is, from
left to right and from the top downwards, the letters of the first section
of the text are inscribed in the cells disclosed by the apertures, the grille
being in the first position. This is shown in b of figure 47. The grille is
then given % turn clockwise, bringing figure 2 to the top left. If the
grille has been correctly prepared, none of the cells disclosed in the
second grille position on the grid will be occupied by a letter. The letters
of the second section are then inscribed, this being shown in c of figure
47. In d and e of figure 47, the results of inscribing the third and fourth
sections, respectively, are shown. The letters of the cryptogram are
then taken out of the completed grid by following any prearranged route
of transcription. The cryptogram below has been transcribed by follow-
ing down the columns in succession from left to right.

d. To decryptograph such a message, the cipher letters are inscribed
columnwise in a grid 10 by 10 (that is, one composed of 100 cells, 10 per
side) and then the grille applied to the square in four consecutive posi-
tions corresponding to those used in cryptographing. The letters dis-
closed by each placement of the grille are written down as they appear,
section after section.

e. The second manner of employing a revolving grille is merely the
reciprocal of the first. The procedure followed in the first method to
decryptograph a message is followed in the second method to crypto—
graph a message; and the procedure followed in the first method to
cryptograph is followed in the second method to decryptograph.

100. Grilles of Other Geometric Forms

Grilles are not limited to square—shaped figures. They may be equi—
lateral triangles, pentagons, hexagons, and so on. Any figure which can
be pivoted upon a central point and which when revolved upon this
pivot can be placed in a succession of homologous positions over a grid
corresponding to the grille will serve equally well. A triangle affords
three grille positions, a pentagon, five, and so on.

101. Polyphase Transposition by Grilles

One grille may be employed to inscribe the letters of the message on
the grid, and a second, and different, grille employed to transcribe them
from the grid to form the final text of the cryptogram. This would con-
stitute a real double transposition method of great complexity. Polyphase
transposition by a series of grilles is of course possible.

112

 

REF ID:A56932

102. Increasing the Security of Revolving Grilles

a. The total number of letters which a grille will exactly encipher is
termed its capacity. If the number of letters of a message is always equal
to the total capacity of the grille, this information is of great aid in solu-
tion by the enemy. For example, a message of 64 letters indicates a grille
8 by 8 with 16 apertures; one of 144 letters, a grille 12 by 12 with 36
apertures, and so on. There are, however, methods of employing a
grille so that it will serve to encipher messages the lengths of which are
greater or less than the capacity of the grille.

b. When the total number of letters is less than the capacity of the
grille, no modification in method of use is necessary. Encipherment of
such a message comes to a close when the last plain-text letter has been
inscribed. In decryptographing such a message, the recipient must strike
out, on the grid upon which he is to inscribe the cipher text, a number of
cells corresponding to the difference between the number of letters of
the text as received and the total capacity of the grille. The location of
the cells to be thus eliminated must be prearranged, and it is best usually
to strike them off from the final positions of the grid.

1529 1301933 57

1.2 2 164346 6207/

I / /
Zj/flV/ZZ 17 u. 31 13 21 '47 34 Z;
/, I /, x 3 32 1.5 7 35 '8 /
QZ/éflf/fl 25 38 11 39 22 36 9 7/
% - /// so 12 26 51 48 10 23 V/
a 27 52 40 28 ’21; 49 37 7/
13 1.1 1!. @V/A/ /////7/

b

 

 

 

 

 

 

 

b

 

 

 

 

 

 

 

 

 

 

 

 

 

 

 

 

 

 

 

 

 

 

Figure 48.

c. When the total number of letters is equal to or greater than the
capacity of the grille, a grid of greater capacity than that of the grille
can be prepared, on which the grille may be positioned several times,
thus forming a large or composite grid composed by the juxtaposition
of the several small grids. If there are a few cells in excess of the actual
number required, these may be struck off from the large grid at pre-
arranged points, for example, from the last column and row, as shown
in b of figure 48. The grille is then placed in its first position in turn on
each of the component grids, then in its second position, and so on. An
example will serve to illustrate. A message of fifty-two letters is to be

113

REF ID:A56932

cnciphered with the grille shown in a of figure 48, the capacity of which
is sixteen letters. The number of letters of the message being greater than
three times sixteen, the composite grid must be composed of four small
grids containing a total of sixty-four cells. Therefore, twelve of these
cells must be eliminated. These are shown in b of figure 48, together with
the number indicating the positions occupied by the letters of the text.

103. Construction of Revolving Grilles

(I. There are several ways of preparing revolving grilles, of which the
one described below is the most simple. All methods make use of cross-
section paper.

b. Suppose a revolving grille with a capacity of 100 letters is to be
constructed. The cells of a sheet of cross-section paper 10 by 10 are
numbered consecutively in bands from the outside to the center, in the
manner shown in a of figure 49. It will be noted that in each band, if n
is the number of cells forming one side of the band, the highest number
assigned to the cells in each band is n —— 1.

c. It will be noted that in each band there is a quadruplication of
each digit; the figure 1 appears four times, the figure 2 appears four
times, and so on. From each receding band there is to be cut out
(n—l) cells: from the outermost band, therefore, nine cells are to be
cut out; from the next band, seven; from the next, five; from the next,
three; and from the last, one cell. In determining specifically what cells
are to be cut out in each band, the only rules to be observed are these:
(1) One and only one cell bearing the figure 1 is to be cut out, one
and only one cell bearing the figure 2 is to be cut out, and so on; (2) as
random a selection as possible is to be made among the cells available
for selection for perforation. In b of figure 49 is shown a sample grille
prepared in this way.

(1. If the side of the grille is composed of an odd number of cells, the
innermost band will consist of but one cell. In such case this central cell
must not be perforated.

e. It is obvious that millions of differently perforated grilles may be
constructed. Grilles of fixed external dimensions may be designated by
indicators, as was done by the German Army in 1915 when this system
was employed. For example, the FRITZ grille might indicate a 10 by 10
grille, serving to cncipher messages of about 100 letters; the ALBERT
grille might indicate a 12 by 12 grille, serving to encipher messages of
about 144 letters, and so on. Thus, with a set of grilles of various dimen-
sions, all constructed by a central headquarters and distributed to lower
units, systematic use of grilles for messages of varying lengths can be
afforded.

f. A system for designating the positions of the perforated cells of a
grille may be established between correspondents, so that the necessity

114

 

REF ID:A56932

for physical transmission of grilles for intercommunication is eliminated.
An example of a possible system is that which is based upon the co—
ordinate method of indicating the perforations. The columns from left
to right and the rows from bottom to top are designated by .the letters
A, B, C, . . . Thus, the grille shown in b of figure 49 would have the
following formula:

ADG; BBEH; CD]; DEG; EACH; FFI; GE; HBDH]; IDG;

JABFI.

 

 

 

 

u: HM'M

 

 

 

 

 

 

~10UIl-‘NUl-‘NWIN
axxnitqu-thowzsu.

coal-INA»

 

Hmu‘hmmqmolp
sol-amwthkucbsit-Jm

 

 

 

 

 

UI‘F‘LANO-‘Hw-hv‘m

 

 

 

mwamu‘buwuo

 

 

bWNI—‘MNHUIO‘Q
memhuqum
HONQO‘UIINWMH

 

 

 

 

@NV“
s\\
\V

\V \V

\V
\
\V
\

 

 

 

\\\\\\
k
§
\\\\
§\
Q

V

 

 

 

 

 

 

 

 

 

 

 

 

 

 

 

\VQkV
N
\V
Q

 

Figure 49.

115

 

 

 

REF ID:A56932

9. Given the formula, the eight corners of the grille can be labeled in
various ways by prearrangement; but the simplest method is that shown
in connection with b of figure 49. Then the initial position of the grille
can be indicated by the number which appears at the upper left-hand
corner when the grille is placed on the grid, ready for use. Thus, position
1 indicates that the grille is in position with the figure 1 at the upper
left-hand corner; position 3, with the figure 3 at the upper left—hand
corner, etc. .

h. The direction of revolving the grille can be clockwise or counter-
clockwise, so that correspondents must make arrangements beforehand
as to which direction is to be followed.

1'. Revolving grilles can be constructed so that they have two operating
faces, an obverse and a reverse face. They may be termed revolving-
reversible grilles. The principles of their construction merely involve a
modification of those described in connection with ordinary revolving
grilles. A revolving—reversible grille will have eight possible placement
indicators; usually positions 1 and 5, 2 and 6, and so forth, correspond
in this obverse—reverse relationship, as shown in figure 43.

j. The principles of construction described above apply also to grilles
of other shapes, such as triangles, pentagons, and so forth.

104. lNonpen‘oral‘ed Grilles

a. All the effects of a grille with actual perforations may be obtained
by the modified use of a nonperforated grille. Let the cells that': would
normally be cut out in a grille be indicated merely by crosses thereon,
and then on a sheet of cross-section paper let the distribution of'- letters
resulting from each placement of the grille on a grid be indicated by
inserting crosses in the appropriate cells, as shown in figure 50.

Grille Grille Position

   

bUNH

Figure 5011. Figure 501).

b. Note should be made of the fact that in figure 50b the distribu—
tion of crosses shown in the third row of cells is the reverse of that
showu in the first; the distribution shown in the fourth row is the reverse
of that shown in the second. This rule is applicable to all revolving
grilles and is of importance in solution.

1:. If the letters of the text are now inscribed (normal manner of
writing) in the cells not eliminated by crosses, and the letters transcribed

116

REF 1D:A56932

from columns to form the cryptogram, the results are the same as though
a perforated grille had been employed. Thus:

 

A I

EIGRIIOLDARDDLTY
Cryptogram:

EWCRA EOLDA RDDAT Y
Figure 506.

d. It is obvious that a numerical key'. may be applied to effect a
columnar transposition in the foregoing method, giving additional
security. I

e. The method is applicable to grilles of other shapes, such as tri-
angles, pentagons, hexagons, octagons, etc.

f. In figure 50c it is noted that there are many cells that might be
occupied by letters but are not. It is obvious that these may be filled with
nulls so that the grid is completely filled with letters; Long messages may
be enciphered by the superposition of several diagrams of the same
dimensions as figure 50c.

105. Rectangular or "Post Card" Grilles

a. The grille shown in figure 51 differs from the ordinary revolving
grille in that (1) the apertures are rectangular in shape, and are greater
in width, thus permitting of inscribing several letters in the cells dis—
closed on the grid by each perforation of the grille; and (2) the grille
itself admits of but two positions with its obverse side up and two with
its reverse side up. In figure 51 the apertures are numbered in succes—
sion from top to bottom in four series, each applying to one position of
the grille; the numbers in parentheses apply to the apertures when the
grille is reversed; the numbers at the corners apply to the four positions
in which the grille may be placed upon the grid.

17. One of the ways in which such a grille may be used is to write the
first letter of the text at the extreme left of the cell disclosed by aperture
1, the second letter, at the extreme left of the cell disclosed by aperture 2,
and so on. The grille is retained in the same position and the 17th letter
is written immediately to the right of the lst, the 18th immediately to the
right of the 2d, and so on. Depending upon the width of the aperture,
and thus of the cells disclosed on the grid, 2, 3, 4 . . . letters may' be
inserted in these cells. When all the cells have been filled, the grille may
then be placed in the second position, then the third, and finally, the
fourth.

117

 

 

 

 

REF ID:A56932

'1 if?! its):
29 I: Eli]
1:11—65 39)::
oz 6 3° (lg-gal
is E

 

 

 

 

““0 (99»:
“)9: 9(41
199)“: 1° 43
11 43 (mm
(gs-)3; 12 44
13 45 (5:9)]:
14 ‘6 (29m
%:§ 15 47
{a}; 16m)
if L!
l s
Figure 51.

c. Another way in which the grille may be used is to change the
position of the grille after the 16th letter has been inserted, then after
the 32d, 48th, and 64th ; the 65th letter is then inserted to the righfl of
the lst, the Slst, to the right of the 17th, and so on until the grid is com-
pleted.

d. Whole words may, of course, be inserted in the cells disclosed by
the apertures, instead of individual letters, but the security of the latter
method is much lower than that of the former.

e. The text of the grid may be transcribed (to form the cryptogram)
by following any prearranged route.

1‘. The successive positions of a post card grille may be prearranged.
The order 1, 2, 3, 4 is but one of 24 different sequences in which it may
be superimposed upon the grid.

g. A modification of the principles set forth in paragraph 103, dealing
with the construction of revolving grilles, is applied in the construction
of rectangular or “post card” grilles. Note the manner in which the
cells in a of figure 51 are assigned numbers; homologous cells in each
band receive the same number. In a of figure 52 there are three bands,

118

 

REF ID:A56932

numbered from 1 to 8, 9 to 16, and 17 to 24. Then in each band one
and only one cell of the same numbered set of four cells is cut out. For
example, if cell 1a is selected for perforation from band 1 (as indicated
by the check mark in that cell), then a cross is written in the other three
homologous cells, 1b, c, and d, to indicate that they are not available for
selection for perforation. Then a cell bearing the number 2 in band 1
is selected, for example, 2c, and at once 2a, b, and d are crossed ofil as
being ineligible for selection, and so on. In c of figure 52 is shown a
grille as finally prepared, the nonshadcd cells representing apertures.

_h. The grille, c of figure 52, is a “six—column” one, that is, the cells
form six columns. It is obvious that grilles with any even number of
columns of cells are possible. The number of apertures in each band
should be equal and this number multiplied by the number of bands and
then by 4 should equal the capacity of the grille. In the case of the one
shown in c of figure 52, the capacity is 8 by 3 by 4 or 96 cells; this is
the same as is obtained merely by multiplying the height (in cells) by the

1

 
   
     
  
   
     

  

Band

 
 

_Wl/V/Il-W/l/l/ll
/// /7//77///7// 7///—
///7/// ’////’///A 7///
/// -7///-j
///7/// //-7////////
/// ;7///.'/// //-'///7
///,7/// //7///A7////7////7///.
///—7///////./7/////////‘
77/ '////.- ////'(/
7///—7/7

  

//A

 

   

:\\:\\:

 

      
 

%\

 

   

       
    
   

’////A'//////

7///A

7/////// ////'////: /////
7///-//// /////’/////

-
//////// /’//
'////)-’///'////// /’/////’////

  
    

 

    

Figure 52.

119

REF ID:A56932

number of columns, 16 X 6 = 96. If four letters are inscribed in each
rectangle, the capacity of the grille in terms of letters is 384. The grid in
this case would, after completion, present 24 columns of letters, to which
a numerical key for a second transposition can be applied in transcrip-
tion to produce the final text of the cryptogram.

106. Indefinite or Continuous Grilles

a. In his Manual of Cryptography, Sacco illustrates a type of grille
which he has devised and which has elements of practical importance.
An example of such a grille is shown in figure 53. This grille contains 20
columns of cells, and each column contains 5 apertures distributed at
random in the column. There are therefore 100 apertures in all, and
this is the maximum number of letters which may be enciphered in one
position of the grille. The plain text is inscribed vertically, from left to
right, using only as many columns as may be necessary to inscribe the
complete message. A 25-letter message would require but 5 columns. To
form the cryptogram the letters are transcribed horizontally from the
rows, taking the letters from left to right as they appear in the apertures.
If the total number of letters is not a multiple of 5, suflicient nulls are
added to make it so. In decryptographing, the total number of letters is
divided by 5, this giving the number of columns employed. The cipher
text is inscribed from left to right and top downwards in the apertures
in the rows of the indicated number of columns and the plain text then
reappears in the apertures in the columns, reading downward and from
left to right. (It is, of course, not essential that nulls be added in the
encipherment to make the length of the cryptogram an exact multiple of
5, for the matter can readily be handled even if this is not done. In de-
cipherment the total number of letters divided by 5 will give the number
of complete columns; the remainder left over from the division will give
the number of cells occupied by letters in the last column on the right.)

 

 

 

 

 

 

1' 31!! V
Jr. Limb.

 

Figure 53a.

120

 

REF ID:A56932

1). Such a grille can assume 4 positions, two obverse and two reverse.
Arrangements must be made in advance as to the sequence in which the
various positions will be employed. That is why the grille shown in figure
53a has the position-designating letter “A” in the upper left-hand corner
and the letter “B” (upside down) in the lower right—hand corner. On the
obverse side of the grille would be the position—designating letters “C"
and I‘D.”

c. Figure 53b shows how a message is enciphered.

Message:
All RECEIVING HEAVY MACHINE GUN FIRE FROM HILL SIX TWO ZERO.

 

 

 

 

Figure 53b.

Cryptogram:
EGIIX FNNEA YTHFL RIRMO IOLWE MERVA ERMAH EGSOA ICUEC NVHIZ.

(The letters E and A in the 10th column are nulls. Columns 11 to 20 are not
used at all, the irregular right-hand edge of the grille merely indicating that this
portion of the grille remains vacant.)

Section V. MISCELLANEOUS TRANSPOSITION SYSTEMS

107. Complex Route Transposition

a. In figure 54 a route for inscribing letters within a rectangle is
indicated by a sequence of numbers. The initial point may be at any of
the four corners of the rectangle, or it may be at any other point, as pre—
arranged. The letters may be inscribed to form the rectangle by following
the route indicated and then transcribed from the rectangle to form the
cryptogram by following another route; or the letters may be inscribed
according to one route and transcribed accordingly to the numerical
route indicated.

b. A variation of the foregoing is that illustrated in figure 55, wherein
the inscription follows the route shown by the arrows. The initial point

121

REF 1D:A56932

of inscription is indicated by the figure 1, and the final point, by the
figure 2.

c. In the foregoing case, the route is a succession of the moves made
by the king in the game of chess; it forms the so-called “king’s tour”,
in which the playing piece makes a complete or reentrant journey cover—
ing all cells of the chessboard, each cell being traversed only once. A
route c0mposed of a succession of moves made by the knight, or the so-
called “knight’s tour”, is also possible, but in order to be practical a grid
with the cells numbered in succession would have to be prepared for the
correspondents, since millions of different reentrant knight’s tours can
be constructed‘ on a chessboard of the usual 64 cells.

a“

 

Figure 54. Figure 55.

108. Transposition of Groups of Letters, Syllables. and Words

There is nothing in the previously described methods which precludes
the possibility of their application to pairs of letters, sets of three or
more letters, or even syllables and whole words. Nor, of course, is their
use limited to operations with plain text they may be applied as second—
ary steps after a substitutive process has been completed (see sec. I, ch.

10).

109. Disguised Transposition Methods

a. The system often encountered in romances and mystery stories,
wherein the message to be conveyed is inserted in a series of nonsig—
nificant words constructed with the purpose of avoiding or evading sus-
picion, is a species of this form of “open” cryptogram involving trans—
position. The “open” or enveloping, apparently innocent text may be
designated as the exle-mal text; the secret or cryptographic text may be.
designated as the internal text. A complicated example of external or
open and internal or secret text is that shown in paragraph 98.

‘Eee BalTW. W. R., Mathematical Recreations and Essays, London, 1928.

122

 

 

w—~ w""'1' ".‘"“" '

 

 

REF ID:A56932

1). Little need be said of the method based upon constructing external
text the letters of which, at prearranged positions or intervals, spell out
the internal text. For example, it may be prearranged that every fourth
letter of the external text forms the series of letters for spelling out
the internal text, so that only the 4th, 8th, 12th . . . letters of the external
text are significant. The same rule may apply to the complete words of
the external text, the n, 2n, 3n, . . . words form the internal text. The
preparation of the external text in a suitable form to escape suspicion
is not so easy as might be imagined, when efficient, experienced, and
vigilant censorship is at work. Often the paragraph or passage containing
the secret text is sandwiched in between other paragraphs added to pad
the letter as a whole with text suitable to form introductory and closing
matter to help allay suspicion as to the presence of secret, hidden text.

c. A modification of the foregoing method is that in which the lst,
3d, 5th, . . . words of a secret message are transmitted at one time or by
one agency of communication, and the 2d, 4th, 6th, . . . words of the
message are transmitted at another time or by another agency of com-
munication. Numerous variations of this scheme will suggest themselves,
but they are not to be considered seriously as practical methods of secret
intercommunication.

(1. Two correspondents may agree upon a specific size of paper and a
special diagram drawn upon this sheet, the lines of which pass through
the words or letters of the internal text as they appear in the external
text. For example, the legs of an equilateral triangle drawn upon the
sheet of paper can serve for this purpose. This method is practicable
only when messages can be physically conveyed by messenger, by the
postal service, or by telephotographic means. Many variations of this
basic scheme may perhaps be encountered in censorship work.

110. Cipher Machines for EFFectin-g Transposition

These may be dismissed with the brief statement that if any exist
today they are practically unknown. A few words are devoted to the
subject in paragraph 147.

123

 

