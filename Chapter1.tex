
\chapter{INTRODUCTION}
 
\section{GENERAL}

\subsection{Scope}

This manual consists of two parts as follows:

\mypara Part one is an introduction to the elementary principles of military
cryptography. In this part a few typical examples of cipher systems and
code systems are presented; the procedure in cryptographing and decryptographing by means of the systems is shown in detail; methods of preparing keys suitable for use in connection with them are illustrated;
errors and their correction are discussed; and finally, a few of the most
important precautions to be observed in safeguarding systems and
cryptograms from enemy cryptanalysts are set forth. Only such considerations as apply to military cryptography are included.

\mypara Part two develops the principles established in part one and treats
of the more advanced systems. Following the presentation sequence of
part one, transposition systems are discussed first, then substitution
systems. Considerable attention is devoted to combined substitution and
transposition methods. Following this is a description of a limited number
of cipher devices and machines, together with a discussion of their
present-day limitations. Finally, code systems are discussed briefly with
special emphasis upon enciphered code systems.

\subsection{Developments in Cryptography}

\mypara Cryptography is by no means a static art or science and viewpoints
are always undergoing change; what is regarded as wholly impracticable
today may, through some unforeseen improvement in technique, become
feasible tomorrow, and it is unwise to condemn a system too hastily.
For example, before World War I, and indeed for the first 2 years of
that conflict, the use of codebooks in the theater of operations was
REF ID:A56932

regarded as wholly impracticable.\footnote{See, in this connection, Friedman, William F., American Army Field Codes in the America Expeditionary Forces During the First World War, Signal Security Service Publication, OCSigO, War Department, Washington, 1942} Colonel Hitt in his Manual for the Solution of Military Ciphers, published in 1916, stated:

The necessity for exact expression of ideas practically excludes the use of cod.
for military work, although it is possible that a special tactical code might be
useful for preparation of tactical orders.

Also, in an official British Army Manual of Cryptography prepared in
1914 is found the following statement:

Codes will first be considered, but as they do not fulfill the conditions required

of a means of secret communication in the field, they need not be dealt with here
at length.

In the 1935 edition of this text the foregoing quotations were immediately succeeded by the following comment:

It need only he pointed out in this connection that today rode methods predominate in the secret communication systems of the military, naval, and diplomatic
services of practically all the large nations cf the world. Nevertheless, it is likely
that within the next decade or two the pendulum may once more swing over to the
other position and cipher methods may again come to the fore, especially if
mechanical and electrical cipher machines are perfected so that their operation
becomes practicable for general use. It is for this reason, if for no other, that the
cryptographer who desires to keep abreast of progress must devote considerable
attention to the more complicated cipher methods of the past and present time,
for with the introduction of mechanical and electrical devices the complexities and
difficulties of these hand—operated methods may be eliminated.

In preparing the revision of this text in 1943, the author found it necessary to say that the forecast he made in 1935 in regard to the rebirth of
cipher methods had been fully justified by the present trend, which is in
a direction away from code and toward cipher methods, because of
important advances made in the field of mechanical and electrical cryptographic mechanisms.

\mypara Modern electrical communication methods and instrumentalities are
finding an increasing need for applications of cryptographic theory and
practice to their efficient operation. For example, in very recent years
there has developed a distinct need for secure methods and means for
distorting voice communications by telephone or radiophone, and
for distorting facsimile transmissions by wire or radiotelegraphy. Teleprinter services permitting direct cryptographic intercommunication by
machines operated from a typewriter keyboard make it desirable to have
means whereby, although the keyboard is operated to correspond to plaintext characters, the latter are instantaneously and automatically
enciphered in transmission and the received signals are instantaneously
and automatically deciphered upon reception at the receiving end. Thus
the printing mechanism at the receiving station records the original
plain-text characters set up on the keyboard at the sending station but
interception Of the signals passing over the line or by radio would yield
only cipher text.

\mypara It is difficult to foresee the specific cryptographic methods which
might some day be useful in connection with developments of the foregoing nature. Progress in the electrical and electronic fields exercises an
important effect upon developments in the cryptographic field. Methods
which today appear to yield a high degree of cryptographic security but
which are impractical for hand operation a few years from now, may be
readily mechanized and become highly practical. On the other hand,
methods which today do provide a high degree of security may, a few
years from now, become obsolete because high-speed electrical analytical
machines have been devised for their rapid solution. Consequently, if
among the many and more or less complex methods set forth herein
certain ones appear to fall outside the realm of what is today considered
practicable, it should be remembered that the purpose in describing them
is to present various basic cryptographic principles, and not to set forth
methods that may with a high degree of probability be encountered in
military cryptography in the immediate future.

\section{TERMINOLOGY}

\subsection{Basic Definitions}

In a study of military cryptography as employed in the U. S. Army,
the following definitions will be useful:

\mypara SIGNAL COMMUNICATIONS. Any means of transmitting messages
in plain or encrypted text other than by direct conversation or mail. A
commander uses signal communications to receive reports of hostile
dispositions and activities, to receive reports of the progress and needs
of subordinate and neighboring friendly units, to send orders to subordinate units, to receive orders from superior units, and to send to higher
and adjacent units information necessary for the coordinated action of
the whole command.

\mypara AGENCY 0F SIGNAL COMMUNICATION. The organization, teams, and
personnel necessary to perform operational duties pertaining to signal
communications.

\mypara MESSAGE. Used in its broadest sense in Department of the Army
and other Official publications, the term “message” includes all instruc-
tions, reports, orders, documents, photographs, maps, or other information, transmitted by means of signal communication. In this manual,
however, the term “message” implies instructions, reports, orders, and
similar communications usually transmitted by electrical means.

\mypara MEANS OF SIGNAL COMMUNICATION. A medium (including equipment) used by an agency for transmitting messages. There are two
dozen or more different means; the most important, so far as this manna
is concerned, are—

\begin{enumerate}
\item \textit{Wire}:

Telephone.

Telegraph.

Teletypewriter.

Facsimile (picture or photo transmission).

\item \textit{Radio}:

Radiotelephone.

Radiotelegraph.

Radio teletypewriter.

Radio facsimile.

\end{enumerate}

\mypara WRITER. The person who actually prepares and signs the message
blank. The writer may be the originator or his officially designated
representative.

\mypara ORIGINATOR. The authority who orders the message written and
sent. The commander may delegate this authority to one or more subor-
dinates. Officers assigned as members of a unit’s general staff are

assumed to have been so designated. .

\mypara TIME or ORIGIN. The time shown on a message by the writer to
indicate the hour and minute when he completed its writing

\mypara ADDRESSEE. The authority (organization, office, or person) to whom
a message is directed by the originator.

\mypara MESSAGE CENTER. That signal communication agency of a head-
quarters, or echelon thereof, which is charged with the receipt, routing
and delivery of all official messages except: those which are transmitted
directly from the originator to the addressee by means of a personal
agent, or telephone or teletypewriter provided for his personal use; mail
handled by military or civil postal services, and purely local messages.

\mypara COMMUNICATIONS CENTER. One or more agencies of signal communication equipped to receive, route, and transmit official messages. A
communications center may be established at a point fixed or mobile.

\subsection{Cryptology and Secret Communication}

\mypara Secrecy of intercommunication in military operations is of the
utmost importance. Need for it has been recognized from the earliest
days of organized warfare. That branch of knowledge which treats the
production, use and solution of the means and methods. of secret\footnote{ Throughout this manual the term "secret" will be used in its ordinary sense as given in the dictionary. Whmever the designation is used in the more restricted sense of the security classification as defined in AR 380—5, it will be so indicated. There are in current use the classifications, Restricted, Confidential, Secret, and Top Secret, listed in ascending order of degree.} communications is called cryptology.


\mypara Intercommunication may be conducted by any means susceptible of
ultimate interpretation by one of the five senses, but those most com-
monly used are visual or auditory. Aside from the use of simple visual
and auditory signals for intercommunication over relatively short dis-
tances, the usual method of intercommunication involves, at one stage
or another, the act of writing, speaking over a telephone, or of drawing
or taking a picture.

\mypara To preserve secrecy of intercommunication by telephone, there are
means and methods of disguising the electrical currents in telephony so
that the messages or conversations can be understood only by persons
provided with the proper equipment. The same thing is true of secrecy
in the electrical transmission of pictures, drawings, maps, etc. However,
this manual is concerned only with secrecy of intercommunication by
means of messages conveying in written words the thoughts, orders,
reports, etc., of the originator to the addressee.

\mypara Writing may be either \textit{visible} or \textit{invisible}. In the former, the characters are inscribed with ordinary writing materials and can be seen with
the naked eye; in the latter, the characters are inscribed by'means or
methods which make the writing invisible to the naked eye. Invisible
writing is done with certain chemicals called invisible, sympathetic, or
secret inks which have the property of either being initially invisible to
the naked eye or becoming so after a short time. In order to make writing
done with secret inks visible, special processes must usually be applied.
There are also methods of producing writings which is invisible because
the characters are of microscopic size. These methods usually require
either special photographic apparatus or very delicate mechanical instruments called micropantographs, by means of which ordinary writing may
be copied in extremely reduced size. Magnifying lenses must be used to
make such writing visible to the naked eye.

\mypara Invisible writing and visible writing prepared in a form unintel-
ligible in the language in which it is written, constitute secret writing.
Both of these forms of secret writing have their uses in military com—
munications, but this manual deals only with visible secret writing.

\subsection{Plain Text and Encrypted Text}

\mypara A visible message which conveys an intelligible meaning in the
language in which it is written, with no hidden meaning, is said to be in
plain text. A message in plain text is called a plain-text message, a clear—
text message, or a message in clear.

\mypara A visible message which conveys no intelligible meaning in any
language is said to be in encrypted text. Such a message is termed a
cryptogram.

\mypara A visible message may convey an intelligible meaning which may
not be the real meaning intended. To quote a simple example of a mes-
sage containing a secret or hidden meaning, prepared with the intention
of escaping suppression by censors in war time, the sentence “Son born
today" may mean “Three transports left today.” Messages of this type
are in encrypted text and are said to be in open code. Although occa-
sionally useful in espionage and counter-espionage, secret communication
systems of this sort are impractical for field military use, and will not
be dealt with further in this manual.

\mypara The term “correspondents” is used in this manual to designate per-
sons who exchange messages with each other. Between the originator
and the addressee there may be persons who actually write and handle
the messages, who convert the plain texts into cryptograms, or who
reconvert the cryptograms into plain texts. The originator and the
addressee may also do this work but in the U. S. Army such work is
usually done by special personnel who act as agents of the correspondents.

\mypara The term “enemy” is used in this manual to designate all persons
who obtain messages or copies of messages not intended for them.

\subsection{Cryptography, Cryptographing. and Decryptographing}

\mypara Cryptography is that branch of cryptology which treats of various
means and methods for rendering plain text unintelligible and recon-
verting unintelligible text into plain text or the application thereof.

\mypara To cryptograph\footnote{ Compare the terms "cryptography," "cryptogram," and "cryptograph" with the terms “telegraphy,” "telegram," and "telegraph."} (encrypt) is to convert a plain-text message into
a cryptogram by following certain rules agreed upon in advance by
correspondents, or furnished them or their agents by higher authority.
The process of cryptographing a message produces a cryptogram.

\mypara To decryptogroph (decrypt) is to reconvert a cryptogram into the
equivalent plain-text message by a direct reversal of the cryptographing
process; that is, by applying to the cryptogram the key used in crypto-
graphing the plain text.

\mypara A person skilled in the art of cryptographing and decryptographing,
or one who has a part in making a cryptographic system is called a
cryptographer; a clerk who cryptographs and decryptographs, or who
assists in such work, is called a cryptographic clerk.

\subsection{Codes, Ciphers, and Enciphered Code}

\mypara Cryptographing and decryptographing are accomplished by means
collectively designated as code: and ciphers. Such means are used for
either or both of two purposes: (1) secrecy, and (2) economy or brevity.
Secrecy usually is far more important in military cryptography than
economy or brevity. In ciphers or cipher systems cryptograms are pro-
duced by applying the cryptographic treatment to individual letters of
the plain—text messages, whereas in codes or code systems cryptograms
are produced by applying the cryptographic treatment to entire words,
phrases, and sentences of the plain—text messages. The specialized mean.
ings of the terms code and cipher are explained in detail later.

\mypara A cryptogram produced by means of a cipher system is said to be
in cipher and is called a cipher message, or sometimes simply a cipher.
Such act or operation of cryptographing is called enciphering, and the
enciphered version of the plain text, as well as the act or process itself,
is often referred to as the encipherment. The cryptographic clerk who
performs the process serves as an encipherer. The corresponding terms
applicable to the decryptographing of cipher messages are deciphering,
decipherment, and decipherer. A clerk who serves both as an encipherer
and decipherer of messages is called a cipher clerk.

\mypara A cipher device is an apparatus or a simple mechanism for literal
encipherment and decipherment, usually manually powered; a cipher
machine is an apparatus or complex mechanism for literal encipherment
and decipherment, usually requiring an external power source.

\mypara A cryptogram produced by means of a code system is said to be
in code and is called a code message, or sometimes simply a code. The
text of the cryptogram is referred to as code text. This act or operation
of cryptographing is called encoding, and the encoded version of the plain
text, as well as the act or process itself is referred to as the encodement.
The clerk who performs the process serves as an encoder. The corres-
ponding terms applicable to the decryptographing of code messages are
decoding, decodement, and decoder. A cryptographic clerk who serves
both as an encoder and decoder of messages is called a code clerk.

\mypara Sometimes, for special purposes, the code text of a cryptogram
undergoes a further step in concealment involving an enciphering process,
thus producing what is called a cryptogram in enciphered code, or an
enciphered-code message. Encoded cipher, the cipher text of a crypto-
gram which subsequently undergoes encodement, is also possible but rare.

\mypara In U. S. Army tables of organization and other publications, cipher
clerks and code clerks are cryptographic technicians. They are specifically
trained to encipher, decipher, encode, and decode messages, using
authorized means, equipment, and procedures.

\subsection{General System and Specific Key}

\mypara The total of all the basic, invariable rules followed in cryptograph—
ing a message according to a given method, together with all the agree—
ments, conventions or private understandings drawn up between the
correspondents or their authorized agents or furnished them by higher
authority, constitute the general cryptographic system.

\mypara In the general cryptographic system usually a number, a group of
letters selected at random, a word, a phrase, or a sentence, is used as a
key. The element selected governs the manner in which a cipher device
Or a cipher machine is prepared for the encipherment or decipherment
of a specific message, or it controls the steps followed in cryptographing
a specific message. This element—usually of a variable nature and
changeable at the will of the correspondents, or prearranged for them or
for their agents by higher authority—is the specific key. The specific key
may also involve the use of a set of specially prepared tables, a special
document, or even a book.

\mypara Hereafter, the general cryptographic system will be referred to as
the system, and the specific key, as the key.

\subsection{Cryptanalytics and Cryptanalysis}

\mypara In theory, any cryptographic system except one can be broken down
if enough time and skill are devoted to it, and if the volume of traffic is
large enough. This can be done even if the general cryptographic system
and the specific key are unknown at the start. The exception is the “one
time” system in which the key is used only once and in itself must have
no systematic construction, derivation, or meaning. In military operations
theoretical rules must usually give way to practical considerations. How
the theoretical rule in this case is affected by practical considerations
will be taken up in subsequent portions of this manual.

\mypara That branch of cryptology which deals with the principles, methods,
and means employed in the solution or analysis of cryptograms is called
cryptanalytics.

\mypara The steps and operations performed in applying the principles of
cryptanalytics constitute cryptanalysis. To cryptanalyze a cryptogram is
to solve it by cryptanalysis.

\mypara A person skilled in the art of cryptanalysis is called a cryptanalyst,
and a clerk who assists in such works is called a cryptanalytic technician.

\section{TWO CLASSES OF CRYPTOGRAPHIC SYSTEMS}

\subsection{Transposition and Substitution}

\mypara Technically there are only two distinct types of treatment which
may be applied to plain text to convert it into secret text, yielding two
different classes of cryptograms. In the first, called transposition, the
elements or units of the plain text, whether one is dealing with individual
letters or groups of letters, syllables, whole words, phrases and sentences,
retain their original identities and merely undergo some change in their
relative positions or sequences so that the message becomes unintelligible.
In the second, called substitution, the elements of the plain texfi retain
their original positions or sequences but are replaced by other elements
with difierent values or meanings.

\mypara It is possible to cryptograph a message by a substitution method
and then to apply a tran5position method to the substitution text, or vice
versa. Such combined transposition-substitution methods do not form a
third category of methods. They are occasionally encountered in military
cryptography, but the types of combinations that are sufficiently simple to
,be practicable for field use are very restricted.

\subsection{Letter, Syllable, and Word Methods}

Under each of the two principal classes of cryptograms as outlined in
the preceding paragraph, a further classification can be made with respect
to the nature of the textual elements or units with which the crypto-
graphic process deals. These textual units are (1) individual letters, or
groups of letters in regular sets, and (2) complete words. Methods which
deal with the first type of units are called letter methods, including, when
such is the case, syllable methods; those which deal with the second type
of units are called word methods.

\subsection{Cipher Systems and Code Systems}

It is necessary to indicate that the classification into letter, syllable,
and word methods is more or less arbitrary or artificial in nature, and is
established for purpose of convenience only. No sharp line of demarca-
tion can be drawn in every case, for occasionally a given system may
combine methods of treating single letters, groups of letters, syllables,
whole words, phrases and sentences. When in a single system the
cryptographic treatment is applied to textual units of regular length,
usually single letters or pairs, and is only exceptionally applied to textual
units of irregular length, the system is called a cipher system. Likewise,
when in asingle system the cryptographic treatment is applied to textual
unit-s of irregular length, usually whole words, phrases, and sentences,
and is only exceptionally applied to single letters, pairs, or groups of
letters and syllables, the method is called a code system because it
generally involves the use of a code book. i

 

\section{SECURITY AND TIME ELEMENTS IN CRYPTOGRAPHIC SYSTEMS}

\subsection{Interception, Radio Direction Finding, and Radio Position Finding}

\mypara Messages transmitted by electrical means can be heard and copied
by persons who are not the correspondents or their authorized agents.
Messages transmitted by radio can be manually copied or automatically
recorded by suitably adjusted radio apparatus located within range of the
transmitter. Some messages transmitted over wire lines can likewise be
manually copied or automatically recorded by special apparatus suited
for the purpose. Correspondents have no way of knowing whether or not
radio transmissions are being copied by the enemy, since the unauthorized
copying does not interfere in the slightest degree with signals being
transmitted. Interception of wire traffic is much more difficult than of
radio, mainly because the equipment must be located very near the wire
line, or connected directly to it. The act of listening-in and copying or
recording electrically-transmitted messages by persons other than the cor-
respondents or their authorized agents is called interception. The pur-
pose of interception is to obtain copies of messages transmitted and, by
studying them, to obtain information. In time of war, it must be assumed
that the enemy will intercept all messages transmitted by any signal
communication agency susceptible of interception.

\mypara It is also possible to determine, with a fair degree of accuracy, the
direction of a radio transmitter from a given location and, by establishing
the direction from two or more locations, it is possible to determine the
geographical location of the transmitter. The science which deals with
the means and methods of determining the direction in which a radio
transmitter lies is called radio direction finding; the method of deter-
mining the geographical location of a radio transmitter, by the use of
two or more direction-finding installations, is called radio position
finding.

\mypara Messages may be transmitted by signals with special apparatus
which distort, disguise, or completely hide the signals themselves, so that
the processes of interception and recording the signals are very difficult,
and intercept personnel may not even be aware Of the existence of such
signals. All such methods of transmitting messages fall in the class
designated in this manual as system of secret signaling. Signaling by
means of so-called “black light,” that is, invisible or infra-red light
waves, falls into this category. Methods of disguising or distorting
voice or picture transmissions (par. 2c) require more or less highly-
specialized apparatus for the interception of the signals and their inter—
pretation or recording in recognizable form. As a rule, the signals of
practically all systems of secret signaling can be intercepted and recorded
in a form suitable to making the signals understood by one of the senses,
usually visual or auditory. Ordinarily, this requires special apparatus
but can sometimes be done without the specific apparatus used in forming
or sending the signals, or the “key” used in their distortion, or disguise.

\subsection{Traffic Analysis and Cryptanalysis}

\mypara A great deal of information of military value can be obtained by
studying signal communications without solving the crytographed mes-
sages constituting the traffic. The procedure and the methods used have
yielded results of sufficient importance to warrant the application of 3
special term to this field of study; namely, traffic analysis\footnote{ Which may be abbreviated tranalysis.  }, which is the
study of signal communications and intercepted or monitored traffic for
the purpose of gathering military information without recourse to crypt—
analysis.

\mypara In general terms, traffic analysis is the careful inspection and study
of signal communications for the purpose of penetrating camouflage
superimposed upon the communication network for purposes of security.
Specifically, traffic analysis reconstructs radio communication networks
by: (1) noting volume, direction, and routing of messages; (2) corre—
lating transmission frequencies and schedules used among and within the
various networks; (3) determining directions in which transmitters lie.
by means of radio direction finding; (4) locating transmitters geo—
graphically, by radio position finding; (5) developing the system of
assigning and changing radio call signs; (6) studying all items that
constitute “conversations” or “chat” exchanged among operators on
radio channel.

\mypara From a correlation of general and specific information derived from
these procedures, traffic analysis is able not only to ascertain the geo-
graphic location and disposition of troops and military units (technically
called "order of battle”) and important troop movements, but also to
predict with a fair degree of reliability the areas and extent of imme-
diately pending or future activities. Traffic analysis procedures are fol-
lowed to obtain information of value concerning the enemy, and to
determine what information concerning our own forces is made avail-
able to the enemy through our own signal communications.

\mypara These very important results are obtained without actually'reading
the texts of the intercepted messages; the solution and translation of
messages are the functions of cryptanalysis and not traffic analysis.
However, the cryptanalyst is frequently able to make good use of bits
of information disclosed by traffic analysis such as faults noted in mes-
sage routing and errors in cryptography causing messages to be duplicated
or canceled. Cryptanalysis can provide important information for traffic
analysis, since the solution of messages often yields data on impending
changes in signal communication plans, operating frequencies and schedules, etc. It also yields data'on specific channels, networks, or circuits

which are most productive of intelligence, so that effective control and
direction of intercept agencies for maximum results can be achieved.
\mypara In addition to (1) traffic analysis and (2) cryptanalysis as means of
obtaining information relating to communications, further data may be
obtained (3) by the use of secret agents for espionage, (4) by the cap—
ture and interrogation of prisoners, (5) by the capture of headquarters
or. command posts with records more or less intact, and (6) by treason
or carelessness on the part of personnel who handle communications. Of
these six main sources, traffic analysis and cryptanalysis are the most
valuable. The amount of vital information they furnish cannot be
accurately estimated as it fluctuates with time, place, circumstances,
equipment, and personnel. For most effective operation, the results of
both cryptanalysis and traffic analysis can be fitted together to yield a
unified picture of the communications scheme. Therefore, if all transmitting stations can be located quickly and if all communications can be
intercepted and solved, extremely valuable information concerning
strength, disposition of forces, and proposed moves will be continually
available.

\mypara The facts set forth above are applicable to our own forces as well
as the enemy’s.

\mypara The process of intercepting and copying our own or friendly radio
and wire transmissions for the purpose of detecting and correcting viola-
tions of regulations is called monitoring; it provides increased protection
of our own signal communications.

\subsection{Communication Intelligence and Communication Security}

\mypara Communication intelligence is evaluated information concerning the
enemy, derived principally from a study of his signal communications.
The main components of communication intelligence are as follows:

\begin{enumerate}
\item Interception of signals or messages and forwarding raw traffic
to communication intelligence centers for study.

\item Traffic analysis, including radio direction finding and radio
position finding. (Evaluated information from this source is
often called trafi‘ic intelligence.)

\item Cryptanalysis or solution (and translation, when necessary) of
the texts of the messages.

\item Evaluation of data, that is, analysis of results obtained from the
preceding steps and their correlation, collation, and comparison
with results obtained from other sources of information.
\end{enumerate}

\mypara Communication security is the protection resulting from all meas—
ures designed to deny to unauthorized persons information of value
which may be derived from communications. The main components of
communication are as follows -:

\begin{enumerate}
\item Physical security, that component of communication security
which results from all measures necessary to safeguard classi-
fied communication equipment, and material from access there-
to by unauthorized persons. ,

\item Cryptosecurity, that component of communication security
which results from the provision of technically sound crypto-
systems\footnote{ Cryptosystems may be categorized as literal and nonliteral. This manual is concerned solely with literal or cryptographic systems.  } and their proper use.

\item Transmission security, that component of communication
security which results from all measures designed to protect
transmissions from interception and traffic analysis.
\end{enumerate}

\mypara Further details on the subject of communication security will be
found in JANAP 122(A) Joint Communications Instructions.

\subsection{Time Needed for Cryptanalysis and its Dependent Factors}

\mypara In military operations time is a vital element. The influence or effect
that analysis of military cryptograms may have on the tactical situation
depends on various time factors.

\mypara Of these factors, the following are the most important:

\begin{enumerate}
\item The length of time necessary to transmit intercepted enemy
cryptograms to solving headquarters. This factor is negligible
only when signal communication agencies are properly and
specifically organized to perform this function.

\item The length of time required to organize raw materials, to make
traffic analysis studies and to solve the cryptograms, and the
time required to make copies, tabulate, and record data.

\item The nature of information disclosed by traffic analysis studies
and solved cryptograms; whether it is of immediate or opera-
tional importance in impending action, or whether it is of his-
torical interest only in connection with past action.

\item The length of time necessary to transmit information to the
organization or bureau responsible for evaluating the informa-
tion. Only after information has been evaluated does it become
military intelligence.

\item The length of time necessary to transmit resulting military
intelligence to the agency or agencies responsible for tactical
operations, and the length of time necessary for the agency to
prepare orders for the action determined by the intelligence and
to transmit them to the combat units concerned. The last
sentence under (1) above applies here also.
\end{enumerate}

\mypara Of the factors mentioned in b above, the only one of direct interest
in this manual is the length of time required to solve the cryptograms.
This is subject to great variation, dependent upon other factors, of
which the following are the most important:

\begin{enumerate}
\item The degree of cryptographic security in the system. The degree
of security depends upon the technical soundness of the system
itself. Technical soundness, in turn, determines the resistance
to analysis which the system offers. Cryptographic systems vary
widely in technical soundness, but this manual does not attempt
to demonstrate such variation.
 
\item The adequacy and technical soundness of regulations drawn up
by designers of the cryptographic system for the guidance of
cryptographic technicians who are its actual users.

\item The extent to which cryptographic technicians follow these
regulations and procedures. Security of a good cryptographic
system can be almost completely destroyed by a few crypto-
graphic technicians who fail to observe the regulations, are
careless in their observance, in sheer ignorance commit serious
violations of cryptographic security, or adopt bad technical
habits. As a result, these technicians jeopardize not only their
own lives but the lives of thousands of their comrades.

\item The volume of cryptographic text available for study. As a
rule, the greater the volume of text, the more easily and
speedily it can be solved. A single cryptogram in a given sys—
tem may present an almost hopeless task for the cryptanalyst,
but if many cryptograms of the same system or in the same or
closely related specific keys are available for study, the solution
may be reached in a very short time.

\item The number, skill, and efficiency of organization and coopera—
tion of signal intelligence units assigned to the work. Crypt--
analytic headquarters are organized in units of ascending size,
ranging from a few persons in the forward echelons to many
persons in the rear echelons. Such organization avoids duplica—
tion of effort and, especially in forward areas where spot intel—
ligence is most useful, makes possible the quick interpretation
of cryptograms in already solved systems. In all these units,
proper organization of highly skilled workers is essential for
efficient operation.

\item The amount and character of information and intelligence avail—
able to the cryptanalytic headquarters. Isolated cryptograms ex-
changed between a restricted, small group of correspondents,
about whom and whose business no information is available,
may resist the efforts of even a highly organized, skilled crypt-
analytic office indefinitely. If, however, a certain amount of
such information is obtained, the situation may be entirely
changed. In military operations usually a great deal of collateral
information is available, from sources indicated in paragraph
146. As a rule, a fair amount of more or less definite informa-
tion concerning specific cryptograms is at hand, such as proper
names of persons and places, and events in the immediate past
or future. Although the exchange of information between intel-
ligence and cryptanalytic staffs is very important, the collection
of information derived from an intensive study of already
solved traffic is equally as important because it yields extremely
valuable cryptanalytic intelligence which greatly facilitates the
solution of new cryptograms from the same sources.
\end{enumerate}

\subsection{Degree of Crytographic Security Required of a System for Military Use}

The ideal cryptographic system for military purposes would be a
single, all—purpose system which would be practicable for use not only by
the largest fixed headquarters but also by the smallest troop unit in the
combat area, and which would also present such a. great degree of crypto-
graphic security that, no matter how much traffic became available, all
in the same key, the cryptograms composing this traffic would resist
solution indefinitely. Such an ideal system however, is beyond the realm
of possibility so far as present methods of cryptographic communication
are concerned; in fact, a multiplicity of systems must be employed, each
more or less specifically designed for a particular purpose. Of each such
system, the best that can be expected is that the degree of security be
great enough to delay solution by the enemy for such a length of time
that when the solution is finally reached the information thus obtained
has lost all its “short term,” immediate, or operational value, and much
of its "long term,” research, or historical value.

\subsection{Fundamental Practical Requirements of a Cryptographic System for Military Use}

\mypara Military cryptograms must meet certain fundamental requirements
of a practical nature because of definite limiting conditions in present
military signal communication means and methods.

\mypara These requirements are (1) reliability, (2) security, (3) rapidity,
(4) flexibility, and (5) economy. Their relative importance is in the
order named.

\mypara Reliability is of first importance. Reliability, as applied to a crypto-
graphic system or device, means that the cryptograms produced by‘ the
sending or originating office will be decryptographed promptly, ac-
curately, and without ambiguity by the receiving office; that the crypto—
graphic system, whether a book, machine, or device, will be on hand and
in good working order, available for instant use; and that when used it
can be expected to be operative as Ionglas needed. Simplicity is implied
in reliability; usually, the more simple the system, the more reliable it is.
Security is. the protection afl‘orded by a sound cryptographic system;
rapidity, the speed with which messages can be cryptographed and de-
cryptographed, usually expressed in words or S-letter groups per minute.
The conflicting requirements of security and rapidity vary according to
circumstances. Signal communication personnel must be governed by
general principles, subject to existing circumstances, rather than by rigid
regulations. Maximum security at all times should be the goal, but in
messages exchanged among the higher headquarters some speed may be
sacrificed to meet greater security requirements, while in messages ex-
changed among the lower headquarters security must often- give .way'to
greater speed requirements. For this reason various cryptographic sys-
tems must be available to meet varying types of situations. As to flexi-
bility,'a cryptographic system specifically adapted for a particular usage
cannot serve as an all-purpose system. A codebook designed for front-
line use can hardly serve the needs of a high headquarters! in the rear;
nor can a cryptographic system designed for use by a high headquarters
serve the needs of a small combat unit. As to economy, the simpler-‘th‘e
operations involved, the shorter will be the texts produced, the amount
of time required to produce the cryptographic material, use it, and trans-
mit the messages; and the greater will be the economy.

\mypara Specific requirements which should be met by a cryptographic sys-
tem for general military use are set forth below.

\begin{enumerate}
\item Cryptograms must be in a form suitable for transmission by
standard telegraphic equipment and methods. This requirement
generally eliminates all systems except those which produce
cryptograms composed of characters readily transmitted by a
telegraphic system employing either the Morse or the'itele—
printer alphabet. Cryptographic systems using Arabic numerals
are not so desirable as those using letters because-the-Mofse
signals for numbers are longer, except when "cut” numbers
are used, and are more difficult for the average American tele-
graph or radio operator to handle. Systems which produce
cryptograms composed of mixtures of letters and figures, or
of letters, figures, and punctuation signs, and which must be
transmitted by' Morse telegraphy are unsuited for practical
usage. However, where Such intermixtures are produced auto-
matically by the cryptographic mechanism and. are transmitted,
received, and deciphered automatically, as certain teleprinter
enciphering systems, their use is permissible. In order to be
Suitable for economical Morse telegraphic transmission, the
cryptographic text must be capable of being arranged in regular
sets of characters for these reasons: first, it promotes accuracy
in telegraphic transmission (since an operator knows he must
receive a definite number of characters in each group, no more
and no less) ;' and secondly, cryptanalysis is usually made more
difficult when the length of the words,.phrases, and sentences
of the plain text is not apparent. The usual grouping is in sets
of five characters, although occasionally other groupings may
be made in special circumstances. Such grouping is not necessary
in teleprinter encipherment systems.
\item Regular channels of signal communication carry only a 
limited volume of traffic. Their most efficient operation demands
that the smallest number of characters actually necessary to
convey a given mesSage be transmitted. Therefore, the crypto-
graphic text should be no longer than its equivalent clear text.
In an exceptional ease, the cryptographic text- may be longer
than the equivalent clear text, but a system in which the
cryptographic text is twice the length of the equivalent clear
text is useful only if it is of outstanding merit and suitable for
certain restricted or special use. No system in which the crypto-
graphic text is more than twice the length of the equivalent
clear text is practicable for military usage. Most of the crypto-
graphic systems in current use produce cryptograms which cor-
respond 1n length with that of the original plain—text message
or are somewhat shorter.

\item General requirements of reliability and speed are that the opera-
tions of cryptographing and decryptographing be relatively
simple and rapid. For use in the combat zone, operations must
be capable of being performed under difficult field conditions
and must not require the remembering and application of a
long series of steps or rules. They must be such as to reduce
the mental strain on the operator to a minimum. Complex
processes requiring several distinct steps are not suited to use
in the combat zone, but occasionally systems involving only
two steps, if each step is simple and rapid, may be practicable
for military usage.

\item Cipher devices or machines for field use must be light 1n weight,
rugged in construction, and simple in operations, requiring the
services of only one operator. Requirements to be met by high—
speed cipher machines are too complex to be described in this
manual.

\item The system must be such that errors, which invariably occur in
cryptographic communications, can be corrected easily and
rapidly by cryptographic technicians. A system is impractical
if frequently it is necessary to call for a repetition of the whole
transmission, or for a rechecking of the original crypto-
graphing.
\end{enumerate}

\mypara Only a few of the systems which fullfill at least several of the fore—
going practical requirements are included in this manual.
