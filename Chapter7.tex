\chapter{FUNDAMENTAL RULES FOR SAFEGUARDING CRYPTOGRAMS}

 

\subsection{General}

The rules given in this chapter are to be considered as a general guide
only. Under actual operating conditions much is dependent upon special
situations, and the specific cryptographic systems employed. Therefore,
the particular rules and regulations currently in effect always will take
precedence over those stated herein.

\subsection{Fundamental Rules of Cryptographic Security}

\mypara Failure to observe the fundamental rules of cryptographic security
often makes possible the solution of cryptographic systems by enemy
cryptanalysts. These rules apply to the originators of messages to be
cryptographed as well as to cryptographic personnel. Detailed instructions for the writers of such messages are outside the scope of this
manual. It is, however, desirable to indicate the following points:
\begin{enumerate}

\item Stereotypic phraseology must be avoided, especially at the
beginning and ending of a message. The known or suspected
presence of stereotypic phraseology constitutes the basis of
many methods employed in cryptanalysis; in some cases, indeed,
the only possible method of solution makes use of the presence
of stereotypic phraseologies, or, as they are often called, cribs.
Operating instructions for currently authorized cryptosystems
prescribe the application of measures which effectively reduce
the dangers of stereotypic phraseology to the security of those
systems; however, as an added precaution, routine reports of
all kinds should be sent by agencies of signal communication
not susceptible to interception.

\item Special care must be taken to see that the messages are clear and
concise. If a message is ambiguous or incomplete, unnecessary
confusion results and the accuracy of the cryptographic operation is brought into question.

\item Messages should be shortened by the deletion of unnecessary
words. Conjunctions, prepositions, repetitions of words, and,
especially, punctuation should be reduced to a minimum. When
punctuation is necessary, it should be spelled out, either in full
or in abbreviated form. Numbers should also be spelled out.
Where letters of the alphabet must be used, as in certain symbols designating types of equipment, it may be necessary to
represent these letters by their authorized phonetic equivalents,
where it is essential that there be no possibility of error. Such
spelling out however, should be reduced to a minimum.

\item Authorized abbreviations should be used whenever practicable.

\item Regulations regarding the manner of indicating addresses and
signatures should be carefully followed.

\item Regulations governing the security classification of messages
(Top Secret, Confidential, Restricted) must be observed at all
times.
\end{enumerate}

\mypara \textit{Much of the success which attends the efiorts of cryptanalysts is
based upon ignorance and carelessness on the part of cryptographic personnel}. Rarely are cryptographic blunders the result of willful violation
of instructions; but if cryptographic personnel realize, that, by carelessness or ignorance, their own lives and those of thousands of their comrades are jeopardized, they will be more attentive to rules set up for their
guidance. The most important of these rules are as follows:
\begin{enumerate}
\item \textit{Questionable messages}. Never cryptograph a message which, in
the opinion of the cryptographer, violates any of the provisions
or regulatons relating to the drafting of messages, until the
question has been referred to and passed by someone with
authority to change the message.

\item \textit{Mixing plain and cryptographic text}. Never allow cryptographic
text with its equivalent plain language to appear in a cryptogram, and never mix plain and cryptographic text, except in
messages where such mixtures are specifically permitted. This
includes punctuation and abbreviations of any description. Such
messages afford valuable clues to the enemy. If a message is to
be cryptographed at all, it should be completely cryptographed.

\item \textit{Text of messages}.

\begin{enumerate}[label=\alph*]
\item Never repeat in the clear the identical text of a message once
sent in cryptographic form, or repeat in cryptographic form
the text of a message once sent in the clear. Anything which
will enable an alert enemy to compare a given piece of plain
text with a cryptogram that supposedly contains this plain
text is highly dangerous to the safety of the cryptographic
system. Where information must be given out for publicity,
or where information is handled by many persons, the plaintext version should be very carefully paraphrased before
distribution, to minimize the data an enemy might obtain
from an accurate comparison of the cryptographic text with
the equivalent, original plain text. To paraphrase a message
means to rewrite it so as to change its original wording as
much as possible without changing the meaning of the message. This is done by altering the positions of sentences in
the message, by altering the positions of subject, predicate,
and modifying phrases or clauses in the sentence, and by
altering as much as possible the diction by the use of synonyms and synonymous expressions. In this'process, deletion
rather than expansion of the wording of the message is
preferable, because if an ordinary message is paraphrased
simply by expanding it along its original lines, an expert can
easily reduce the paraphrased message to its lowest terms, and
the resultant wording will be practically the original message.
It is very important to eliminate repeated words or proper
names, if at all possible, by the use of carefully selected
pronouns; by the use of the words “former,” “latter,” “first—
mentioned,” “second—mentioned”; or by other means. After
carefully paraphrasing, the message can be sent in the other
key or code.

\item Never send the literal plain text or a paraphrased version of
the plain text of a message which has been or will be transmitted in cryptographed form except as specifically provided
in appropriate regulations.
\end{enumerate}

\item \textit{Keys}. Never repeat in a different key or system, without paraphrasing, a cryptographed message which has once been trans-
mitted, unless specifically authorized by the appropriate
authority.

\item New cipher keys. Never transmit a new cipher key by means of
a message cryptographed in an old key.

\item Addresses or signatures. Never place cryptographed addresses
or signatures at the beginning or end of the cryptographed text.
Bury them in the body of the message.

\item Identifying information. Include in the address of a cryptographed message only the minimum information necessary for
the message to reach the headquarters for which it is intended.

\item \textit{Replies}. Never reply to a cryptographed message in the clear.

\item \textit{Short titles}. Never use short titles as system or message indicators in cryptographed messages.

\item \textit{Dummy letters and padding}. Never use dummy letters or padding unless their use is specifically authorized.

\item \textit{System indicator}. Never encipher, encode, or disguise in any
way the system indicator, unless specifically authorized.

\item \textit{Notations}. Never place on the cryptographed copy of a
message any notations about the system or the subject matter
of the message.

\item \textit{Work tables}. Never allow unnecessary materials such as
books, documents, or papers to be on the work table during the
process of cryptographing and decryptographing.
\item  \textit{Filing messages}. Never tile cryptographic messages and their
equivalent plain text together. Work sheets must be destroyed
by burning.
\item \textit{Check for accuracy}. Cryptographed messages should be
checked for accuracy by decryptographing the message before
transmission. Whenever practicable, this should be done by a
cryptographer other than the one who originally cryptographed
the message.
\item) \textit{Safeguarding material}. Observe all rules of physical security
established to safeguard the cryptographic material and message
translations. Utmost care should be taken to prevent the loss
or unauthorized sight of the codes or lists of cipher keys in
use. It is possible to photograph an entire code in two or three
hours. Mere continued possession of the cryptographic material
is, therefore, no absolute guaranty that it has not been compromised by photography or some other method of reproduction.
The only absolute assurance of its not having been compromised is that it has never left the possession of the person into
whose care it has been entrusted or the safe in which it is kept
when not in use. Even if knowledge that a code or cipher has
been compromised follows immediately after such compromise,
the amount of time and the difliculties involved in notifying all
concerned and distributing new cryptographic material are so
great that serious damage is caused by the delay and interruption in communication, not to speak of the danger resulting
from the enemy’s reading the most recent messages in the compromised system.

\item \textit{Reporting compromise}. Finally, it must be realized that the
compromise or capture of cryptographic material is a most
serious matter. If there is any reason to suspect that such
material or related documents have been compromised, higher
authority should be notified by the fastest means possible. Not
only is such material available to the enemy for reading current
and old messages, but also the cryptanalytic data afforded
thereby become most useful in working on similar systems to
replace the compromised one. The failure to notify higher
authority promptly, if compromise is suspected, may jeopardize
the lives of thousands of soldiers and is therefore more serious
than permitting compromise to take place, if it could have been
avoided. Regulations for reporting compromise should be care-
fully observed at all times.

\end{enumerate}
