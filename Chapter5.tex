\chapter{COMPARISON OF CODE AND CIPHER SYSTEMS}
\subsection{Advantages and Disadvantages of Each Type of System}
\mypara From the viewpoint of purely military cryptography, a comparison
of the advantages and disadvantages of each method can be made only
between systems suitable for each of the following three general categories:
\begin{enumerate}
\item High—security, or “high—grade,” systems for cryptographic
intercommunication among the largest military units and the
highest echelons of command.

\item Medium-security, or “medium—grade,” systems for crypto-
graphic intercommunication among the intermediate units and
echelons of command.

\item Low-security, or “low-grade,” systems for cryptographic inter-
communication among the small units and the lowest echelons
of command.
\end{enumerate}

\mypara The principal factors to be taken into account in comparing code
and cipher methods in cryptographic communication are reliability,
security, rapidity, flexibility, and economy.
\begin{enumerate}

\item \textit{Reliability}. Reliable cipher machines made possible by modern
engineering and cryptographic techniques satisfy all or a
majority of these factors to a great degree, and such machines
are now used in the U. S. Army for these high—grade systems.
Although the machines are complex, their reliability can be
assured by having properly trained personnel to operate and
maintain them. Accuracy is also one of the elements of
reliability and a good cipher machine can yield a higher degree
of accuracy or completeness of text in cryptographic communi-
cation than can a code system. A mistake in one or two code
groups may obscure, alter, or render unintelligible the meaning
of a whole message, but in cipher systems, often wrong
letters may be corrected, or missing letters may be supplied,
by the context. It must be remembered, however, that in some
cipher systems a single error of a fundamental type, such as
using the wrong key or the wrong “setting,” may prevent the
deciphering of the message.

\item \textit{Security}. If reliability were the only or the most important
factor, code would be preferable to cipher for all echelons of
command, because the simplicity of a code book is to be preferred to the complexity of a large cipher machine. But unen-
ciphered code is not sufficiently secure for the communications
of the highest echelons and headquarters. If encipherment must
be added as a second step in the cryptographic process, it
practically destroys the simplicity features of a code system;
unless the enciphering method is fairly complex, it adds little
security. In a properly designed cipher machine, embodying
sound cryptographic principles based upon a thorough knowl-
edge of cryptanalytic principles, the single—step-encipherment
process can yield cryptograms of very great security. In a good
code system, however, the solution of one or even of several
messages does not entail the immediate breakdown of the
entire system, with the consequent ability to read all messages,
as is usually the case in a cipher system.\footnote{A good cipher system may be compared to a library housed in a large structure of many rooms with all doors and all windows securely locked. If an intruder can force an entry into the structure, he will find a master key which will open all the locks and give him access to all the books in the library. A good code system (especially a two-part code) may be compared to a library housed in a similar structure, but no two locks are alike and no master key is available or can be made. Therefore, the lock on each door must be worked at patiently as a separate problem. Thus, although the intruder may force his way into one room, this gives him access to only a small part of the library; in order to read all the books, he must force his way into each room, which takes much time, since each lock presents a separate and special problem.} Codebooks, of course,
can be rendered useless by compromise. Actual possession for
a long period of time is not necessary; methods of rapid
photography may be applied and a book of several hundred
pages copied in a few minutes.

\item \textit{Rapidity}. The speed with which a cipher n-iachine equipped with
a typewriter keyboard can be operated leaves even simple,
unenciphered code far behind in the matter of rapidity.
Flexibility. Complete flexibility would permit cryptographing
the originator’s own language without change necessitated by
the limitations that exist in all but the most extensive codebooks.
Thus, a cipher machine is much more flexible than a code and
can be used for all sorts of messages; whereas, in a code containing words, phrases, and sentences prepared for a specific
type of communication, rewording the original text as written
by the originator is often necessary, if the words, phrases, and
sentences in the codebook are to be used; otherwise the original
wording must be encoded word by word, or even syllable by
syllable.

\item \textit{Economy}. Whether expressed in terms of money or manpower, cipher systems are more economical than code systems
for high-echelon communications. Code text is usually shorter
than the equivalent plain text, because it is condensed or
abbreviated, but a single clerk operating a rapid cipher machine
can turn out 10 to 15 times as much work as one operating a
code system; furthermore, codes must be prepared, printed,
and distributed. These steps take much time and labor and are
often performed under considerable difficulty. A continuously
operative code compilation section must be maintained to replace
codes as fast as they become compromised by continued use, or
by capture. The handling of the manuscript and proofs in printing entails the necessity of ever watchful secrecy; and finally,
the prompt and thorough distribution of codes to all who must
use them is sometimes very difficult, especially where the distribution must be made over an extensive territory. Therefore,
for high—echelon cryptographing communications, ciphers are
more economical than codes, but the economy factor is least
important.
\end{enumerate}

\mypara It is clear that high-grade systems should include all or as many as
possible of the five factors listed in b above; moreover, the advantages
afforded by good cipher machines make cipher systems more desirable
than code systems for the high-grade cryptographic systems required by
high-echelon cryptographic intercommunication. In addition, secondary
or “back—up" systems must be provided so that in case of machine or
power failure, there will be available some means for cryptographic communication. Finally, emergency systems must be provided for crypto—
graphic communication when neither apparatus nor codebooks can be
employed.

\mypara Medium-grade cryptographic systems for intercommunication
among intermediate echelon commands must meet almost the same severe
requirements as systems for intercommunication among high—echelon
commands. Here again, cipher machines are preferable to code. The
machines may not be so large or complex, but if the same basic crypto—
graphic system is employed by both the high—grade and the medium—grade
machine, many advantages are noted. The problems of manufacture,
maintenance, instruction of personnel in the operation of the system,
distribution of keying data, etc., are simpler if they are basically the
same for both types of machines. Moreover, it is possible for a message
from an intermediate command to be deciphered by a high—echelon com-
mand, and vice versa, without using a second cryptographic system. For
these reasons, cipher machines are widely used in the U. S. Army for
medium-grade cryptographic communication and, in addition, certain
manual systems requiring simple types of apparatus are also used. These
may serve also as the secondary or “back—up” systems required for the
high-echelon cryptographic communications.

\mypara \begin{enumerate}
\item Even in the so—called “low-grade” systems, cipher machines are
serving the purposes for which field codes were formerly sup—
plied. Converter M—209—( ) is a small, mechanical cipher
machine widely used in the U. S. Army for communications
within the small combat units. If properly used, it yields crypto-
grams of considerable security. It is a complicated device; it
has no keyboard, and is slow in operation. Despite its reduced
size and weight, this device is not convenient for use in frontline areas, nor is it suitable for use in voice communication by
small radio—telephone equipments such as the “walky—talky” or
"handy-talky” sets.

\item Manual or hand-operated cipher systems are also unsuitable for
such purposes. The processes of enciphering and deciphering by
means of such systems require very close mental attention to
avoid errors; the more secure methods are hopelessly slow and
the faster ones are not secure, in comparison with the security
that a small, frequently-changed two-part code yields.

\item Practical experience indicates that in messages of very small
tactical units, in certain types of air-to-air or air-to—ground
communications, and in certain forms of messages where the
subject matter is highly stereotypic, as in weather reports and
fire-control observations, code is often preferred over cipher.
In all these cases, speed must give way to security; size and
weight of equipment are important factors; simplicity of
operation undcr battle conditions is vital, which eliminates
methods requiring much training and concentrated attention.
Also, if code is properly prepared, one or two code groups may
express a command or a report that would require many groups
of cipher text. Small codes meet the requirements in all these
respects, and for this reason, code is still used to some extent
in the U. S. Army, especially in the forward areas.
\end{enumerate}

\subsection{Fundamental Assumption of Military Cryptography}

It has been seen that every good cryptographic system combines two
more or less separate .and distinct elements: a basic or unchangeable
method or process, which is termed the general system; and a specific or
variable factor which controls the steps under the general system and is
termed the specific key. The secrecy of any military cryptographic system
must be entirely dependent upon the specific key because it must be
assumed that the enemy is in full possession of all the details concerning
the general system. This assumption is warranted by the whole history
of military cryptography and is based upon the two following considerations which all experienced cryptanalysts regard as valid. In the first
place, in military cryptography there are more prolific sources from which
to obtain information concerning cryptographic methods than there are
in the isolated methods used by private individuals. In fact, by one
means or another, the enemy can sooner or later come into possession of
full information regarding the general cryptographic system. In the
second place, within a very short time the number of messages available
for study becomes so great, and the inevitable blunders in the handling
of communications have become so numerous that a solution by detailed
study can always be made by the enemy, with a consequent disclosure of
the general system. If a cryptographic system adopted for military use
were such that messages in that system could be solved easily without
the specific keys applicable to the messages once the underlying methods
became known, the entire system would have to be changed, a new system
devised, and thousands of persons in the military service trained in its
operation. This, of course, would be impracticable. It is assumed that
the enemy has knowledge of the general cryptographic system, its cipher
devices, instruments, or machines. Only cryptographic documents which
are given a limited distribution can be kept secret from the enemy, but
they can be kept secret only for a variable length of time before they
must be changed. These changes, as a rule, do not affect their method of
usage. In cipher sytems, the specific key must be susceptible of easy
and rapid changes by prearrangement between correspondents. In
systems for use by secret agents or very small military parties in the
theater of operations, the key may be an easily remembered word, phrase,
sentence, or number; it must not require the carrying of written notes on
the person. In systems for use by commanders of large and intermediate
or even small headquarters in the theater of operations, the specific key
may be in the form of written memoranda, paper tapes, and the like.
Generally, the specific key must be the same throughout a given period
of time for all the members of an intercommunicating network, or at
least only a very limited number of specific keys must be in simultaneous
effect; otherwise confusion and delay are inevitable. As a consequence
of this requirement, the enemy may intercept a good many messages all
in the same specific key. A cryptographic system for military use must
conform to all requirements of practicability set forth in section IV,
chapter 1, and to the foregoing section concerning the specific key; this
system must be such that it is practically impossible for the enemy to
solve any message quickly enough to make the information obtained of
real or immediate value in the tactical situation, even though he is in full
knowledge of the general method of the system, possesses the cipher
device or apparatus, if used, and may have available for study 1,000 or
more cryptograms sent on the same day. There is no single cryptographic
system yet known which fully meets all these requirements, and in order
to provide the necessary degree of security for a large army several
different types of ciphers and cipher machines, as well as small codes
for front line use, must be employed simultaneously.
